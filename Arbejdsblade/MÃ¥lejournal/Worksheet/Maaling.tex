\chapter{Measurement Journal}

\section*{Near-ground path loss }
To test the near-ground path loss, the loss from the transmitting antenna (Tx) to the receiving antenna (Rx) will be measured, where the antennas are placed at different heights and with different distances between them. From worksheet link budget it is known that by measuring the power received from a reference signal, the path loss(PL) can be found if the gains of the antennas and system loss are known.

\subsection*{Equipment}
\textbf{Antennas}
\begin{itemize}
\item 2x Patch antennas 2.58 GHz
\item 2x Patch antennas 858 MHz
\item 2x Monopole antennas 2.58 GHz
\item 2x Monopole antennas 858 MHz
\end{itemize}

\textbf{Measurement equipment}
\begin{itemize}
\item Signal generator (AAU nr: 33376)
\item Spectrum analyser (AAU nr: 56915)
\end{itemize}

\textbf{Setup equipment}
\begin{itemize}
\item 2x High stands 2.5 m
\item 2x Low stands (Bags) 0.14 m
\end{itemize}

\textbf{Extra}
\begin{itemize}
\item Computer to note down measurements
\item Wagon
\item 2x Clamps
\item Extension cord +35 m
\item SMA male/male connector
\item SMA cables (2.5 m SUCOFLEX\_104, 1.5 m SUCOFLEX\_104 and 1 m rg223\_U 
\end{itemize}

\subsection*{Measurement points}

\textbf{Heights of antennas}
The Tx and Rx are set in one of these heights for each measurement. 

\begin{itemize}
\item 0.04 m
\item 0.14 m
\item 0.36 m
\item 2.02 m
\end{itemize}

All combinations of different heights of the two antenna are measured, except those combination where the Rx is place higher than the Tx, as these will give the same measurement, as if the two height where switched. This is done because it is assumed the PL is the same independently of which antenna is the receiver and transmitter.

\textbf{Distance between antennas}
The distance between the antennas are set to one of these distances for each measurement.

\begin{itemize}
\item 1 m
\item 2 m
\item 4 m
\item 8 m
\item 15 m
\item 30 m
\end{itemize}

\textbf{Total measurement points per setup}

For each combination of heights, each distance is measured. With 10 height combinations and 6 distances, this gives 60 measurement points. In each point there will measured 10 times, which gives a total of 600 measurements per setup.

\textbf{Different setups}
The measurements are performed in two locations an outdoor empty parking lot and and indoor empty school gym. Furthermore, as horizontal and vertical polarization do not act the same, each antenna set will be tested at both polarizations, given 16 setups in total.

\subsection*{Setup}
The setup is made of a stationary station and a moveable station. The moveable station is placed, so the setup has the right distance between the antennas.

\textbf{Stationary station}
At the stationary station the signal generator is placed and tuned to the frequency of the antenna set used and a amplitude on 0 dBm. A high stand and a low stand are placed beside the signal generator. The Tx is set a the highest measurement point on the high stand and is connected to the signal generator. The antenna is set to face the Rx and have the same polarization as the Rx.

\textbf{Moveable station}
At the moveable station the spectrum analyser is placed and tuned to the antenna set's frequency. The spectrum analyser needs power but is moved with the moveable station, therefore an extension cord is needed, to power it. A high stand and a low stand are placed beside the signal analyser. The Rx is set a the highest measurement point on the high stand and is connected to the spectrum analyser. The antenna on the high stand is placed, so it is facing the Tx and has the same polarization. The distance between the two stands is set to the smallest of the measurement distances.


\subsection*{Procedure}
%To minimize the time to make the measurement, a procedure is chosen, so that the time used on changing the position of the antennas between each measurement point is lowered.
%As changing the heights from 200 or 34 cm take longer than moving the moveable station, all measurements at these heights will be taken subsequently of each other, if possible. 
%The order of measurements, that is used is:
The procedure for a single setup is:

\begin{enumerate}
\item Tx is placed at 2.02 m
\begin{itemize}
\item Rx is placed at 2.02 m, all distances are measured
\item Rx is placed at 0.36 m, all distances are measured
\item Rx shifting between 0.14 m and 0.04 m, all distances are measured
\end{itemize}
\item Tx is placed at 0.36 m
\begin{itemize}
\item Rx is placed at 0.36 m, all distances are measured
\item Rx shifting between 0.14 m and 0.04 m, all distances are measured
\end{itemize}
\item Tx/Rx shifting between 0.14/0.14 m, 0.14/0.04 m and 0.04/0.04 m , all distances are measured
\end{enumerate}

At each measurement point, there is taken 10 measurements, which is read from the spectrum analyser and written down into an excel file.

All this is done for each polarization (vertical and horizontal) on all 4 sets of antennas at both locations.

\subsection*{Results}
The results will be the total system loss, from the signal generator to the spectrum analyser. The total system loss is a combination of:

\begin{itemize}
\item Cable loss, from the signal generator to the Tx.
\item Antenna gain, in the Tx.
\item Path loss, form the Tx to the Rx.
\item Alignment loss, from the polarization of the Tx and Rx.
\item Antenna gain, in the Rx.
\item Cable loss, from the Rx to the spectrum analyser.
\end{itemize}

As can be seen in worksheet link budget.
