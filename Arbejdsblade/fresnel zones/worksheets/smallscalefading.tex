\chapter{Small Scale Fading}

Small scale fading refers to small changes occurring both in the amplitude and phase over small distances, distances as small as half a wavelength. This means that the phase and amplitude keep changing, over small distances. The change in the amplitude and phase, is caused by multipath propagation and the Doppler shift.  
\\
\\
Multipath propagation is a result of reflecting objects and scatters in the propagation path.  
The received signal is the sum of all the signals arrived in different paths. Constructive summation of signals will cause an increase in the signal, while an de-constructive summation of the signals will cause an reduction of the signal, this all depending on how the phases of the signals add together. Deep fading may occur, which is a big loss of signal, where it could be that the two signals are out of phase. As the signals reflect and have an an direct way, this means that the signals have different lengths to arrive to the receiver, this causes the receiver to see multiple copies of the signal at different times of arrival.  


\subsection{Rayleigh Fading}
%Rayleigh fading does not consider the direct ray, which means that it only takes into account the reflection.
The Rayleigh distribution is good to use as a model for the channel propagation, when there is no strong line of sight path from the transmitter to the receiver. This could be when there is NLOS, where the transmitter is hidden behind a big building, where the transmitter cannot see the receiver, this will give a weaker line of sight signal.

\subsection{Rician Fading}
Rician fading distribution is used as a model for the channel propagation, when the direct line of sight is dominant, and there a few, reflected signals. 

%takes into account the LOS component to the  Rayleigh model, so it takes into account both the direct and the reflected signal. Rician fading is used when there is dominant non-stationary signal that is dominant. 




%%http://rfmw.em.keysight.com/wireless/helpfiles/n5106a/about_fading.htm

%%http://www.keysight.com/upload/cmc_upload/All/07-24-03-Fading-Schmitz_1mb-notes.pdf?&cc=DK&lc=dan

%%http://www.eee.hku.hk/~sdma/elec6040_2008/Part%204-Communications%20over%20wireless%20channels.pdf

%%http://www.iitg.ernet.in/scifac/qip/public_html/cd_cell/chapters/a_mitra_mobile_communication/chapter5.pdf

%%https://books.google.dk/books?id=OL7aBwAAQBAJ&pg=PA161&lpg=PA161&dq=small+scale+fading+rician&source=bl&ots=KPxUNkwLP0&sig=yQ20RMl17oV3mjrjbi30RWP94n8&hl=da&sa=X&ved=0ahUKEwjAhfKorI_QAhXCjCwKHbmXAtw4ChDoAQhGMAU#v=onepage&q=small%20scale%20fading%20rician&f=false

%%http://www.ee.fju.edu.tw/pages/032_faculty/sclin/lecture/wireless_comm/ch5.pdf

%%http://www.slideshare.net/nagasirisha756/3-thewirelesschannel2

%%http://www.planetanalog.com/document.asp?doc_id=527327