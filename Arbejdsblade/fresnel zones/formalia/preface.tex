\chapter*{Preface}\label{ch:forord}%\addcontentsline{toc}{chapter}{Forord}

This bachelor project in Communication Systems has been carried out during the spring of 2016, by a group of three students from Electronics and IT at Aalborg University.

The project regards the development of a communication link between a satellite in space and a ground station, as well as  the development of a prototype implementation of a software defined radio to be used in the link.
%This report and the prototype described in the following is made by a group of three students on sixth semester "Electronics and IT" at Aalborg University.

A general knowledge of electronic engineering and more specifically communication systems is needed to read the report.  

References to sources are of the APA-style, i.e. of the type [\emph{source name/author's surname, year, optional page number}]. Sources are listed in a bibliography at the end of the report, and PDFs, datasheets etc. can be found in the attached ZIP-file. Figures without a source are made by the project group.% Appendices are listed after the bibliography.

This report is organized in three parts. The first part is a technical analysis that goes in depth with antennas, a link budget of the communication link between a satellite and a ground station at AAU, different suitable modulation types and, finally, a requirements for a prototype of the radio to be used in the communication link is established. \\ In part two, the theory, simulation and implementation of the prototype is described. \\ Part three contains the test of the prototype related to the requirements set up for it in part 1. %Finally, part four contains descriptions of how two created antennas was made and descriptions of three test setups utilized in project and their related results.

Special thanks should be addressed to PhD student Dereje Assefa and associate professor John Hansen for helping with the USRP's and LabVIEW. A thanks to assistant engineer Kristian Bank for helping with tests of the antennas in Starlab and for the great help regarding setting up and borrowing measurement equipment. Furthermore, a thanks goes to associate professors Flemming Bjerre Frederiksen and Carles Navarro Manchón for giving a helping hand in finding reading material.

\vspace{0.5\baselineskip}\hfill Aalborg University, \today
\vfill

%% underskrifts afsnit %%

\vspace{1.5\baselineskip}
\begin{minipage}[b]{0.45\textwidth}
 \centering
 \rule{\textwidth}{0.45pt}\\
  Amalie Vistoft Petersen\\
 {\footnotesize apet13@student.aau.dk}
\end{minipage}
\vspace{1.5\baselineskip}
\hfill
\begin{minipage}[b]{0.45\textwidth}
 \centering
 \rule{\textwidth}{0.45pt}\\
  Rasmus Gundorff Sæderup\\
 {\footnotesize rsader13@student.aau.dk}
\end{minipage}

\noindent\makebox[\textwidth][c]{%
\begin{minipage}[b]{0.45\textwidth}
 \centering
 \rule{\textwidth}{0.5pt}
   Thomas Kær Juel Jørgensen\\
 {\footnotesize tkjj13@student.aau.dk}
\end{minipage}}
