\section{Two Ray Plane Earth}
In contrast to the Friss pathloss model, the Two-ray-ground-reflection path loss model \citep{two_ray}, considers both the direct wave and the reflected ground wave. Also the Two-ray-ground-reflection path loss model does not depend on the frequency, as the Friss pathloss model does. The received power depending on the distance is given in the following Formula:

\begin{equation}
P_r(d) = \frac{P_t G_t G_r h^2_t h^2_r}{L \cdot d^4}
\label{two_ray_model}
\end{equation}

Where $h^2_t$ and $h^2_r$ are the heights of the transmitter and receiver antennas respectively. And L is the system loss. 

If the distance $d$ is less then a critical point $d_{c}$: 

\begin{equation}
d<d_{c}
\label{two_ray_cond}
\end{equation}

Where $d_{c}$ is given as:

\begin{equation}
d_{c} = \frac{4\pi \cdot h_t h_r}{\lambda}
\label{critical_fac_dc}
\end{equation}

If this condition is true then the two-ray model shall not give good results due to oscillations which are caused by the constructive and destructive combination of the two rays. This condition is true for small distances $d$.

%else will the received power theoretically oscillates between
%local maxima of 6dB above free space to $-\infty$ dB at local minima.

%\subsubsection{Power vs distance} 

%\subsubsection{Propagation path}

%The path can be divided into three segments the first being if:

%\begin{equation}
%d<h_{t}
%\end{equation}

%If this condition is true then:

%\begin{itemize}
%The two rays add constructively
%Path loss is slowly increasing 
%\end{itemize}




%LPE = 40 log10(d) $-$ 20 log10(ht) $-$ 20 log10(hr )