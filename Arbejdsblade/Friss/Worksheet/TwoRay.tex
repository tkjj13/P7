\section{Two Ray Plane Earth}
In contrast to the Friss pathloss model, the Two-ray-ground-reflection path loss model \citep{two_ray}, considers both the direct wave and the reflected ground wave. Also the Two-ray-ground-reflection path loss model does not depend on the frequency, as the Friss pathloss model does. The received power depending on the distance is given in the following Formula:

\begin{equation}
P_r(d) = \frac{P_t G_t G_r h^2_t h^2_r}{L \cdot d^4}
\label{two_ray_model}
\end{equation}

Where $h^2_t$ and $h^2_r$ are the heights of the transmitter and receiver antennas respectively. And L is the system loss. 

If the distance $d$ is less then a critical point $d_{c}$: 

\begin{equation}
d<d_{c}
\label{two_ray_cond}
\end{equation}

Where $d_{c}$ is given as:

\begin{equation}
d_{c} = \frac{4\pi \cdot h_t h_r}{\lambda}
\label{critical_fac_dc}
\end{equation}

If this condition is true then the two-ray model shall not give good results due to oscillations which are caused by the constructive and destructive combination of the two rays. This condition is true for small distances $d$.

\subsection{Critical point calculation}
For the measurements done, the condition is tested:

\begin{equation}
d<d_{c}
\label{two_ray_cond_1}
\end{equation}

\subsubsection{For 2m and 2m heights of transmitter and receiver antennas}

For 2m and 2m heights of the transmitter antenna and receiver antenna. This for a frequency of 858MHz:

\begin{equation}
d_{c} = \frac{4\pi \cdot 2m \cdot 2m}{0.3494m} = 143.86m
\label{critical_fac_dc_calc_2_2_858MHz}
\end{equation}

So for 2m and 2m, for all distances from 1m to 30m between the sender and the receiver ,the results when using the Two-ray propagation model, shall experience oscillations which are caused by the constructive and destructive combination of the two rays, for a frequency of 858MHz.

While the the same for 2.58GHz:

\begin{equation}
d_{c} = \frac{4\pi \cdot 2m \cdot 2m}{0.1161m} = 432.94m
\label{critical_fac_dc_calc_2_2_2.58GHz}
\end{equation}

Again the condition is not met, for all distances, for 2m and 2m.


\subsubsection{For 2m and 0.34 heights of transmitter and receiver antennas}
While when the height of the transmitter antenna is 2m while the receiver antenna is placed at 0.34m, also for a frequency of 858Mhz:

\begin{equation}
d_{c} = \frac{4\pi \cdot 2m \cdot 0.34m}{0.3494m} = 24.45m
\label{critical_fac_dc_calc_2_0.34}
\end{equation}

So for 2m and 0.34m for all distances from 1m to 15m, this condition is true while for a distance of 30m this condition is not true.

While the the same for 2.58GHz:

\begin{equation}
d_{c} = \frac{4\pi \cdot 2m \cdot 0.34m}{0.1161m} = 73.60m
\label{critical_fac_dc_calc_2_0.34.58GHz}
\end{equation}

Again this condition is not met for all distances.

0.01, 0.08, 0.34, 2
 


%else will the received power theoretically oscillates between
%local maxima of 6dB above free space to $-\infty$ dB at local minima.

%\subsubsection{Power vs distance} 

%\subsubsection{Propagation path}

%The path can be divided into three segments the first being if:

%\begin{equation}
%d<h_{t}
%\end{equation}

%If this condition is true then:

%\begin{itemize}
%The two rays add constructively
%Path loss is slowly increasing 
%\end{itemize}




%LPE = 40 log10(d) $-$ 20 log10(ht) $-$ 20 log10(hr )