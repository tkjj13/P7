
The results from the measurement campaign are analysed to determine influence of the different test parameters. The influence of the parameters is found from the sample mean across the other parameters along with their 95\% confidence interval. According to \autoref{App1} when changing the distance from one measurement point to the next, there is an increase in PL of around 6 dB across all distances point.
\vspace{2em}
\begin{table}[!htbp]
\centering
\caption{Sample mean PL across all measurements for different distances}
\label{App1}
\begin{tabular}{|c|c|c|c|}
\hline
Distance    & 1 m & 2 m& 4 m\\\hline
PL & (34.7$\pm 1.6$) dB & (41.4$\pm 1.4$) dB & (49.0$\pm 1.7$) dB  \\\hline
\multicolumn{4}{c}{}\\\hline
Distance	&8 m& 15 m& 30 m\\\hline
PL &	(57.3$\pm 2.1$) dB & (66.1$\pm 2.5$) dB & (72.3$\pm 2.3$) dB \\\hline
\end{tabular}
\end{table}
\newpage


In \autoref{App2}, the sample mean PL for all height setups is shown. It should be noted that it is expected that the confidence interval is larger as fewer samples is used to estimate the mean PL across the heights compared to the other parameters. It is further seen from \autoref{App2} that, by lowering the heights, the PL increases between 3 dB and 5 dB.

\begin{table}[H]
\centering
\caption{Sample mean PL across all measurements for different height combinations}
\label{App2}
\resizebox{\linewidth}{!}{
\begin{tabular}{|c|c|c|c|c|}
\hline
Tx \textbackslash Rx & 0.04 m & 0.14 m & 0.36 m & 2.02 m \\
\hline
0.04 m & (63.7$\pm 5.2$) dB & (60.7$\pm 5.1$) dB & (55.4$\pm 4.7$) dB & (52.4$\pm 3.8$) dB\\
\hline
0.14 m & (60.7$\pm 5.1$) dB & (58.1$\pm 5.2$) dB & (53.4$\pm 4.5$) dB & (50.2$\pm 3.2$) dB\\
\hline
0.36 m & (55.4$\pm 4.7$) dB & (53.4$\pm 4.5$) dB & (49.0$\pm 2.9$) dB & (47.6$\pm 4.8$) dB\\
\hline
2.02 m & (52.4$\pm 3.8$) dB & (50.2$\pm 3.2$) dB & (47.6$\pm 4.8$) dB & (44.4$\pm 3.1$) dB\\
\hline
\end{tabular}}
\end{table}


The rest of the test parameters (Environment, Antenna type and Polarization) sample mean PL is seen in \autoref{App3}. It can be seen on \autoref{App3} that by changing the parameters, the mean PL varies between 2 dB and 4 dB.

\begin{table}[H]
\centering
\caption{Difference bewteen setups of different parameters.}
\label{App3}
\resizebox{\linewidth}{!}{
\begin{tabular}{|c|c|c|c|c|c|}\hline 
\multicolumn{2}{|c}{Environment} & \multicolumn{2}{|c}{Antenna type} & \multicolumn{2}{|c|}{Polarization}\\\hline
Gym & (52.4$\pm 1.8$) dB & Monopole & (55.6$\pm 2.0$) dB &Vertical & (51.8$\pm 1.9$) dB \\\hline
Parking lot & (54.6$\pm 2.2$) dB & Patch & (51.4$\pm 2.0$) dB &Horizontal & (55.1$\pm 2.1$) dB \\\hline
\end{tabular}}
\end{table}


It should be noted that even though there only are a minimum of two setups tested for each parameter, these setups are quite different, so if the parameters have any influence on the PL, it should be apparent. 






%This is expected for the placement and antenna type parameters, as the antenna type should not effect the PL, and the differences for the placements is found in the reflection, which have been taking into consideration for each placement. The polarization is also taking into consideration with the reflection and the antenna gains.
 

