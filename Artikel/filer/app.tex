To make a simplify model, there will be looked at the different test parameters, to see if some of these parameters have significant less influence on the path loss than others of the test parameters.

In the other models, the distance between the antennas and the height of the transmitting and receiving antenna is used in most of them. These parameters also influence the PL significantly. When changing the distance from from measurement point to the next measurement point, there is a rise in PL on 6 dB across all measurement.

\begin{tabular}{|c|c|c|c|c|c|}
\hline
   & 1 to 2m & 2 to 4m & 4 to 8m & 8 to 15m & 15 to 30m \\
\hline
PL & 6.54 & 7.36 & 8.44 & 8.79 & 6.12 \\
\hline
\end{tabular}

The height of the antennas influence the PL by a higher loss near ground.

\begin{tabular}{|c|c|c|c|c|}
\hline
Tx Rx & 0.01m & 0.08m & 0.34m & 2.00m \\
\hline
0.01m & 
\hline
0.08m & 
\hline
0.34m & 
\hline
2.00m & 
\hline
\end{tabular}

Compared to these two parameters, the rest of the test parameters (Placement, Antenna type and Polarization) do have a less impact, as all three parameters only have two different setups and the differences between these setups is 2.23, -4.18, 3.32 dB respectively, and there is no significantly tendencies found in the data for these parameters. This is expected for the placement and antenna type parameters, as the antenna type should not effect the PL, and the differences for the placements is found in the reflection, which have been taking into consideration for each placement. The polarization is also taking into consideration with the reflection and the antenna gains.

So for the test parameters, there will only be seen at the distance between and height of antennas. The other parameters is much smaller in influence and taking into account in other ways, so will not be looked at.
