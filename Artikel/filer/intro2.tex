\subsection{Path loss models}\fixme{mere om conditions for approximationerne evt i footnotes}

In terms of calculating the PL, different propagation models can be applied, to calculate the power received, given different conditions. The Friss free space path loss (FSPL) equation  \eqref{simple_friss} calculates the power received, given only free space loss \cite{Chong}.

\begin{equation}
P_r = \frac{P_t G_t G_r}{L} \left(\frac{\lambda}{4 \pi d}\right)^2
\label{simple_friss}
\end{equation}

where $P_{r}$ and $P_{t}$ are the power received and transmitted respectively, $G_t$ and $G_r$ are the gains in the transmitting and receiving antenna respectively and $L$ is the system loss. While $\lambda$ is the wavelength of the transmitted signal, $d$ is the distance between the transmitting and receiving antenna. The model above given in \eqref{simple_friss} is not the the complete FSPL, as it assumes no losses due to polarization mismatch, pointing error, and matching between the system and antennas \cite{full_friss}. \\

The FSPL only accounts for the direct wave between the receiver and the transmitter. A model that also accounts for the single point reflected wave is the two-ray-ground-reflection path loss model (TRPL) \cite{two_ray}. %and is given in the following Equation:

\begin{equation}
P_r(d) = \frac{P_t G_t G_r }{L}\left(\frac{h_t h_r}{d^2}\right)^2
\label{two_ray_model}
\end{equation}

where $h_t$, $h_r$ are the heights of the transmitting and receiving antenna respectively. \\


A way to decide between which models to use the FSPL or the TRPL, is based on the following condition \cite{two_ray}. 

\begin{equation}
d < d_{c}
\label{two_ray_cond}
\end{equation}

where $d_{c}$ is called the cross over distance and is given as:

\begin{equation}
d_{c} = \frac{4\pi \cdot h^2_t h^2_r }{\lambda}
\label{two_ray_cross_dis}
\end{equation}  
%\eqref{two_ray_cond}, 

The condition above \eqref{two_ray_cond}, is given as the TRPL model considers two waves, a direct and a reflected wave, this model predicts interferences, which are caused by the constructive and destructive combination of the two waves. If the condition is true use  use TRPL, while if false use FSPL.   %The cross over distance $d_c$ can be found using \eqref{two_ray_cross_dis} \cite{two_ray}. 
\\

%\begin{equation}
%P_r(d) = \frac{P_t G_t G_r }{L}\left(\frac{h_t h_r}{d^2}\right)^2
%\label{two_ray_model}
%\end{equation}

%where $h_t$, $h_r$ are the heights of the transmitting and receiving antenna respectively. \\
%As the TRPL model considers two waves, a direct and a reflected wave, this model predicts interferences, which are caused by the constructive and destructive combination of the two waves. The interference is dominant if \eqref{two_ray_cond} is true\cite{two_ray}.


%this gives the following condition:


%a condition is introduced given as:

%\begin{equation}
%d < d_{c}
%\label{two_ray_cond}
%\end{equation}

%where %$d_{c}$ is called the cross over distance and is given as:

%\begin{equation}
%d_{c} = \frac{4\pi \cdot h^2_t h^2_r }{\lambda}
%\label{two_ray_cross_dis}
%\end{equation}  

However when placing the antennas at ground level, the TRLP predicts that the power received is zero, which is not the case.
The reason is that another wave becomes a factor. This wave is called the surface wave \cite{Chong}. The surface wave assumes a minimum effective height of the antennas as can be seen in \eqref{surface_wave}. 

\begin{equation}
P_r=\frac{P_t G_t G_r }{L}\left(\frac{h_0}{d}\right)^4
\label{surface_wave}
\end{equation}

where $h_0$ is the minimum effective height of the antennas. The condition in terms of when the surface wave becomes an dominant factor is, when the antennas are located near the ground, thus when $h_{r,t}$ $<$ $\lambda$ \cite{Chong}. When this condition is true $\Delta$ $\approx$ 0, thus the TRLP and FSLP can be neglected.    
While when both of the antennas are elevated at least one $\lambda$ above ground, or 5-10 $\lambda$ above water \cite{Chong}, the surface wave can be neglected. 
The three above mentioned models \eqref{simple_friss}\eqref{two_ray_model}\eqref{surface_wave} originate from the same model \eqref{ground_wave}, but is approximations given different circumstances. The complete model is called the ground wave model \cite{Chong,Bullington}. %\fxnote{Hvornår skal denne model bruges}
%A combination of the above mentioned models, shall give the power received, with respect to the direct wave, reflected wave and the surface wave this can be expressed as the ground wave \cite{Chong} and is given in the following equation:


\begin{equation}
P_r=P_0 \left|\frac{E}{E_0}\right|^2 
\label{ground_wave}
\end{equation}
where
\begin{equation}
\frac{E}{E_{0}}=[\underbrace{\underbrace{1}_{FSPL}+R\text{e}^{j\Delta}}_{TRPL}+\underbrace{(1-R)A\text{e}^{j\Delta}}_{surface}]
\label{ground_wave_EE0}
\end{equation}

Here $P_{0}$ is equivalent to FSPL, $R$ is the reflection coefficient of the surface, $\Delta$ is the phase difference between the direct and reflected wave and $A$ is the surface attenuation factor.

The three models are obtained by taking \eqref{ground_wave_EE0} and setting the terms outside their\footnote{Burde der stå " the three models"} respective brackets to 0 and using appropriate approximations.

From the complete ground wave model given in \eqref{ground_wave_EE0}, the TRLP can be simplified to \eqref{two_ray_model}, 
when $\Delta$/$2$ $<$ $\pi$/$10$ $\Rightarrow$ $\sin$ $\frac{\Delta}{2}$ $\approx$ $\frac{\Delta}{2}$. This is given for $d$ $<$ $(20h_{t}h_{r}/\lambda)$, if this condition is true the simplified TRLP \eqref{two_ray_cond} can be applied.  
  
  
% the direct wave is in use, and therefore the FSPL will be multiplied by 1. While if the TRPL it is indicated as $P_{0}$, and is multiplied with the reflected wave factor, where the rest is set as zero. 




%where ${direct}$ represents the friis transmission equation, ${reflected}$ represents two-ray-ground-reflection path loss model, and  ${surface}$ represents the surface wave. 

%In order to also get the 

%The two-ray-ground-reflection path loss model,  


%If introduced condition is true the model predicts oscillations which are caused by the constructive and destructive combination of the two rays.


%a parameter called the cross over distance $d_{c}$ is introduced, and is given as:






%Another model which takes into account the free space loss and the reflection   