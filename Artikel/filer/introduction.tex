
In the future it is likely that more and more wireless sensor networks (WSN) will appear, such WSNs could be used to monitor traffic flow, or home power consumption. Examples could also include industrial or military uses. Many of such networks may be placed close to or directly in the ground, such an example is to monitor the traffic flow, where sensors could be placed in the ground. In terms of military use such sensors could also be placed in the ground to detect incoming enemies. In such networks both power efficiency as well as reliability is key to maximise the performance. To estimate those a reliable model for the path loss (PL) is needed. %Such a path loss model indicates the power received from the sensors to the base station, based on various parameters like polarization loss, cable loss etc. When applying typically models like the FSPL, problems occur when placing the antenna to close to the ground. 
These problems still needs to be investigated further to effectively estimate the PL. Many of the earliest works only focus on frequencies below 30 MHz \cite{Bullington}, and states that the complexity increases as frequency increases. Most of the more recent works focus only on equal height for receiver and transmitter or neglect a verification of the purposed model. 


%The problem about calculating the path loss for antennas located near ground has been looked on.
