
In the future it is likely that more and more wireless sensor networks (WSN) will appear. Such WSNs could be used to monitor traffic flow or home power consumption. Examples could also include industrial or military uses. Many of the WSN nodes may be placed close to or directly on the ground. An example is to monitor the traffic flow, where sensors could be placed at ground level. In terms of military use, the nodes could also be placed at ground level at different locations to detect incoming enemies. In such networks both power efficiency as well as reliability is key to maximise the performance. A characteristic for these scenarios is that the antennas are placed near-ground, meaning that the ground has a significance for the path loss (PL). To estimate those, a reliable PL model is needed.
Multiple models for estimating the PL for near-ground scenarios exists. Two articles found to have significant content on the subject are: Bullington's \cite{Bullington}, which has a theoretical approach to the problem and suggests a thorough, but complicated solution to the problem and Chong and Kim's \cite{Chong}, who tries to simplify the model proposed by Bullington, but here the focus is mainly on irregular terrain and only one height has been used for the measurements.

This article focuses on exploring existing PL models and their strengths and weaknesses compared to each other. This is done based on a measurement campaign that focuses on near-ground PL but covers several other parameters, which could influence the PL. This concludes in an extended PL model, which removes some of the weak points in existing models.

% but many of the earliest works only focus on frequencies below 30 MHz \cite{Bullington}, and states that the complexity increases as frequency increases. Most of the more recent works focus only on equal height for receiver and transmitter or neglect a verification of the purposed model. 


%The problem about calculating the path loss for antennas located near ground has been looked on.
