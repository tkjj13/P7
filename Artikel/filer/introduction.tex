
In the future it is likely that more and more wireless sensor networks (WSN) will appear. Such WSNs could be used to monitor traffic flow or home power consumption. Examples could also include industrial or military uses. Many of the WSN nodes may be placed close to or directly on the ground. An example is to monitor the traffic flow, where sensors could be placed at ground level. In terms of military use, the nodes could also be placed at ground level at different locations to detect incoming enemies, where all of these nodes will have to report back to the base station, indicating if enemies are near. In such networks both power efficiency as well as reliability is key to maximise the performance. To estimate those, a reliable path loss (PL) model is needed.
There are multiple models made for estimating the PL at low transmitter (Tx) or receiver (Rx) heights. Two articles found to have significant content on the subject are: Bullington's \textit{Radio Propagation at Frequencies Above 30 Megacycles}, which has a theoretical approach to the problem and suggest a thorough, but complex solution to the problem and Chong and Kim's \textit{Surface-Level Path Loss Modelling for Sensor Networks in Flat and Irregular Terrain}, which tries to simplify the model proposed by Bullington, but here the focus is mainly on irregular terrain and only one height has been used for the measurements.

This article focuses on validating existing PL models, and their strengths and weaknesses compared to each other. This is done based on a measurement campaign that focuses on near ground PL, but covers several other parameters which could influence the PL. This concludes in an extended PL model, which removes some of the weak points in the existing models.

% but many of the earliest works only focus on frequencies below 30 MHz \cite{Bullington}, and states that the complexity increases as frequency increases. Most of the more recent works focus only on equal height for receiver and transmitter or neglect a verification of the purposed model. 


%The problem about calculating the path loss for antennas located near ground has been looked on.
