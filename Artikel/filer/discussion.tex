
By looking at \autoref{model_comparison} it can be seen that be making the simplification of the GWPL, not only loses the model applicability, as it is now necessary to calculate the conditions before assessing the the PL, but also the inaccuracy of the prediction increases. This is due to some weak points of the simplifications made. It can be seen that FSPL actually predicts quite well inside its valid region, but it is also the model that has second fewest valid points in the measurement campaign, due to the assumption of no multipath. Where TRPL has the highest valid region of the simplified models has a rather high MSE, this is due to its lack of ability to account for the surface wave, as seen from \eqref{two_ray_model} when the heights go to zero so does the power received. The NSPL is worst off, lacking in both applicability as well as accuracy, the weak point here is primarily the need of $z$. This leaves all model with crucial weak points. This paper will now propose a model that accounts for some of these weak points. 


In terms proposing a PL model six parameters is considered distance, height of Tx, height of Rx, antenna type, polarization and environment. In Appendix it is shown that there is very little correlation between signal level and antenna type, polarization and environment compared to the distance and heights, therefore those factors are deemed unnecessary to consider, which leaves only the distance and heights as needed to be considered in our proposed model. This is in expected as only the NSPL has some dependency on polarization and environment. 

This means that when designing a PL model it is reasonable build it around the existing models. The aim is to make a simple version of the GWPL model, has high applicability and high accuracy, but is still simple to calculate compared to GWPL. Therefore the

that involves the simplified PL models, which does not require complex data processing. However when using the NSPL model \eqref{surface_wave} it is necessary to make measurements to find $z$, the aim is then to make an average of $z$ values, that is only valid for the two surfaces the measurement have been made, which can be used as the $z$ value. The purposed model shall consist of the simplified TRPL \eqref{two_ray_model} added together with the NSPL \eqref{surface_wave}.

\begin{equation}
Our \quad model = TRPL + NSPL
\end{equation}

The reasoning behind adding exactly these two models is that the TRPL and NSPL overlap each other, given there conditions, where if the purposed model included FSPL, then the TRPL is not valid where FSPL is valid. 


%both takes into account the direct and reflected wave, while the NSPL, takes into account the surface wave. This is given for the conditions that the two models have to uphold \eqref{cond_surface}, \eqref{two_ray_cond}.

%As both the TRPL and NSPL have their conditions these same conditions will be applied to the our purposed model. 
When comparing the MSE and coverage are of our purposed model and the GRPL, where the MSE of our purposed model is 65.42, while the coverage is 65.00$\%$. The MSE of the GRPL is still better.  \\


%The measurements at the smallest distances between the transmitter and the receiver, will suffer from not being in the Far Field, this could indicate why some of the measurements behave as they do. \\

%Also other measurements at different distances have a weird jump this could involve an environmental factor, as also the measurments have been done over three days, at the parking lot. 
     
