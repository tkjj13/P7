

In terms proposing a PL model all six parameters from the measurement campaign have been considered. From \autoref{App1} and \autoref{App2} it is shown that the distance and heights are very influential and should therefore form the base of the proposed model. The proposed model is designed around the existing models, as the theory for those are still firm. The aim is therefore to make a model that is simple, compared to the GWPL model, but has high applicability and high accuracy compared to the other PL models. 


The base of the model is chosen to be the ATRPL as it already has the highest applicability of the models except for the GWPL and a simple structure compared to the GWPL. The weak point in the ATRPL is its inaccuracy at very low heights which is a problem as the focus are near-ground scenarios. This could be compensated by the NSPL, the reasoning behind this is that NSPL accounts for the surface wave part that has been set to zero in the approximation of the ATRPL. Together these model will account for the all three waves in the GWPL. These can however not be simply added together as it is the power of the waves that can be added and not the respective PL. To find the predicted $P_r$ from the models, \eqref{two_ray_model} and \eqref{surface_wave} are inserted into \eqref{link_bud} and added finally the PL is isolated again. A more simple way to express this is seen in \eqref{proposedModel1}.


\begin{align}
Proposed\; PL\; model &= \left(ATRPL^{-1} + NSPL^{-1}\right)^{-1} \label{proposedModel1}\\
L_p &= \frac{d^4}{h_t^2 h_r^2+h_0^4} \label{proposedModel2}
\end{align}

This model still suffers from the weak point of NSPL's dependency upon $z$ which is expressed in \eqref{h_0}. However, as only the magnitude of $z$ is needed this can be guestimated in most cases and in more critical cases it can be measured using the method described in \cite{Kim}. From the measurements the average magnitude of $z$ across environment and polarization is 0.8122. To find out the performance of the model it is compared to the other models using this average of $z$. First to analyse the applicability of the model it has the same applicability as the ATRPL model which means it can be used if \eqref{two_ray_cond} is true else FSPL should still be used. This can be seen from \autoref{ourModel2} and \autoref{Models62} as the intersection between the two aforementioned models.


\begin{figure}[H]
\centering
% This file was created by matlab2tikz.
%
%The latest updates can be retrieved from
%  http://www.mathworks.com/matlabcentral/fileexchange/22022-matlab2tikz-matlab2tikz
%where you can also make suggestions and rate matlab2tikz.
%
\definecolor{mycolor1}{rgb}{0.00000,0.44700,0.74100}%
\definecolor{mycolor2}{rgb}{0.85000,0.32500,0.09800}%
\definecolor{mycolor3}{rgb}{0.92900,0.69400,0.12500}%
\definecolor{mycolor4}{rgb}{0.49400,0.18400,0.55600}%
\definecolor{mycolor5}{rgb}{0.46600,0.67400,0.18800}%
%
\begin{tikzpicture}

\begin{axis}[%
width=2.8in,
height=\myvar,
at={(0.758in,0.481in)},
scale only axis,
xmode=log,
xmin=0.9,
xmax=31,
xlabel=Distance (m),
xminorticks=true,
xmajorgrids,
xminorgrids,
ymin=20,
ymax=100,
ylabel=Path loss (dB),
ymajorgrids,
axis background/.style={fill=white},
title style={font=\bfseries},
legend style={legend cell align=left,align=left,draw=white!15!black},
legend pos = north west
]

\addplot [color=mycolor1,mark size=2.5pt,only marks,mark=asterisk,mark options={solid}]
  table[row sep=crcr]{%
2.1541736234575	40.5286184873639\\
2.7641389255969	41.7634386176021\\
4.43175631099003	43.3569736894665\\
8.22438228683468	49.22358923555\\
15.1208618801972	58.5868821697177\\
30.060613167399	67.725648643832\\
};
\addlegendentry{Measured PL};

\addplot [color=mycolor2,solid]
  table[row sep=crcr]{%
1	31.1115179430228\\
1.29292929292929	33.3430134440292\\
1.58585858585859	35.1168070992565\\
1.87878787878788	36.5890729354301\\
2.17171717171717	37.8475832493839\\
2.46464646464646	38.9466105778464\\
2.75757575757576	39.9220669918869\\
3.05050505050505	40.7989529102148\\
3.34343434343434	41.5953739265862\\
3.63636363636364	42.3248640664175\\
3.92929292929293	42.9978060775859\\
4.22222222222222	43.6223396865725\\
4.51515151515152	44.2049645137105\\
4.80808080808081	44.7509531054816\\
5.1010101010101	45.264641613445\\
5.39393939393939	45.7496391916429\\
5.68686868686869	46.2089819480987\\
5.97979797979798	46.6452481855302\\
6.27272727272727	47.0606460546034\\
6.56565656565657	47.4570811839289\\
6.85858585858586	47.8362095366818\\
7.15151515151515	48.1994792048672\\
7.44444444444444	48.5481638082528\\
7.73737373737374	48.8833894437239\\
8.03030303030303	49.2061566242012\\
8.32323232323232	49.5173582850141\\
8.61616161616162	49.8177946744222\\
8.90909090909091	50.1081857537082\\
9.2020202020202	50.3891815905317\\
9.49494949494949	50.6613711230658\\
9.78787878787879	50.9252895920871\\
10.0808080808081	51.1814248768192\\
10.3737373737374	51.4302229230174\\
10.6666666666667	51.6720924150277\\
10.959595959596	51.9074088147627\\
11.2525252525253	52.136517867826\\
11.5454545454545	52.3597386589774\\
11.8383838383838	52.5773662847132\\
12.1313131313131	52.7896741991299\\
12.4242424242424	52.9969162798597\\
12.7171717171717	53.199328653229\\
13.010101010101	53.3971313115477\\
13.3030303030303	53.5905295503075\\
13.5959595959596	53.7797152488309\\
13.8888888888889	53.9648680143974\\
14.1818181818182	54.1461562069475\\
14.4747474747475	54.3237378590187\\
14.7676767676768	54.4977615035086\\
15.0606060606061	54.6683669201317\\
15.3535353535354	54.8356858099672\\
15.6464646464646	54.9998424062559\\
15.9393939393939	55.1609540285398\\
16.2323232323232	55.3191315863387\\
16.5252525252525	55.4744800377779\\
16.8181818181818	55.6270988079186\\
17.1111111111111	55.7770821709656\\
17.4040404040404	55.9245196000324\\
17.6969696969697	56.069496087713\\
17.989898989899	56.2120924403367\\
18.2828282828283	56.3523855484555\\
18.5757575757576	56.4904486358333\\
18.8686868686869	56.6263514889533\\
19.1616161616162	56.760160668845\\
19.4545454545455	56.8919397068421\\
19.7474747474747	57.0217492857095\\
20.040404040404	57.149647407435\\
20.3333333333333	57.2756895488449\\
20.6262626262626	57.3999288060896\\
20.9191919191919	57.5224160289408\\
21.2121212121212	57.6431999457502\\
21.5050505050505	57.7623272798382\\
21.7979797979798	57.8798428580096\\
22.0909090909091	57.9957897118245\\
22.3838383838384	58.1102091721996\\
22.6767676767677	58.2231409578586\\
22.969696969697	58.3346232581061\\
23.2626262626263	58.4446928103564\\
23.5555555555556	58.5533849728113\\
23.8484848484848	58.6607337926463\\
24.1414141414141	58.7667720700345\\
24.4343434343434	58.8715314183094\\
24.7272727272727	58.9750423205423\\
25.020202020202	59.0773341827885\\
25.3131313131313	59.1784353842344\\
25.6060606060606	59.2783733244589\\
25.8989898989899	59.3771744680074\\
26.1919191919192	59.4748643864588\\
26.4848484848485	59.5714677981531\\
26.7777777777778	59.6670086057337\\
27.0707070707071	59.7615099316476\\
27.3636363636364	59.8549941517351\\
27.6565656565657	59.9474829270312\\
27.9494949494949	60.0389972338908\\
28.2424242424242	60.1295573925447\\
28.5353535353535	60.2191830941809\\
28.8282828282828	60.3078934266443\\
29.1212121212121	60.3957068988359\\
29.4141414141414	60.4826414638918\\
29.7070707070707	60.5687145412132\\
30	60.653943037416\\
};
\addlegendentry{FSPL};

\addplot [color=mycolor3,solid]
  table[row sep=crcr]{%
1	11.1879572974509\\
1.29292929292929	15.6509482994637\\
1.58585858585859	19.1985356099183\\
1.87878787878788	22.1430672822656\\
2.17171717171717	24.6600879101731\\
2.46464646464646	26.8581425670981\\
2.75757575757576	28.8090553951792\\
3.05050505050505	30.562827231835\\
3.34343434343434	32.1556692645777\\
3.63636363636364	33.6146495442404\\
3.92929292929293	34.9605335665772\\
4.22222222222222	36.2096007845503\\
4.51515151515152	37.3748504388264\\
4.80808080808081	38.4668276223687\\
5.1010101010101	39.4942046382954\\
5.39393939393939	40.4641997946912\\
5.68686868686869	41.3828853076028\\
5.97979797979798	42.2554177824657\\
6.27272727272727	43.0862135206121\\
6.56565656565657	43.8790837792632\\
6.85858585858586	44.637340484769\\
7.15151515151515	45.3638798211397\\
7.44444444444444	46.061249027911\\
7.73737373737374	46.7317002988531\\
8.03030303030303	47.3772346598077\\
8.32323232323232	47.9996379814336\\
8.61616161616162	48.6005107602498\\
8.90909090909091	49.1812929188217\\
9.2020202020202	49.7432845924689\\
9.49494949494949	50.2876636575369\\
9.78787878787879	50.8155005955795\\
10.0808080808081	51.3277711650438\\
10.3737373737374	51.8253672574401\\
10.6666666666667	52.3091062414607\\
10.959595959596	52.7797390409309\\
11.2525252525253	53.2379571470573\\
11.5454545454545	53.6843987293602\\
11.8383838383838	54.1196539808318\\
12.1313131313131	54.5442698096652\\
12.4242424242424	54.9587539711248\\
12.7171717171717	55.3635787178634\\
13.010101010101	55.7591840345007\\
13.3030303030303	56.1459805120203\\
13.5959595959596	56.5243519090673\\
13.8888888888889	56.8946574402002\\
14.1818181818182	57.2572338253004\\
14.4747474747475	57.6123971294427\\
14.7676767676768	57.9604444184226\\
15.0606060606061	58.3016552516687\\
15.3535353535354	58.6362930313398\\
15.6464646464646	58.9646062239172\\
15.9393939393939	59.286829468485\\
16.2323232323232	59.6031845840827\\
16.5252525252525	59.9138814869611\\
16.8181818181818	60.2191190272425\\
17.1111111111111	60.5190857533365\\
17.4040404040404	60.8139606114701\\
17.6969696969697	61.1039135868314\\
17.989898989899	61.3891062920787\\
18.2828282828283	61.6696925083163\\
18.5757575757576	61.945818683072\\
18.8686868686869	62.2176243893119\\
19.1616161616162	62.4852427490954\\
19.4545454545455	62.7488008250896\\
19.7474747474747	63.0084199828244\\
20.040404040404	63.2642162262753\\
20.3333333333333	63.5163005090951\\
20.6262626262626	63.7647790235846\\
20.9191919191919	64.009753469287\\
21.2121212121212	64.2513213029057\\
21.5050505050505	64.4895759710818\\
21.7979797979798	64.7246071274246\\
22.0909090909091	64.9565008350544\\
22.3838383838384	65.1853397558046\\
22.6767676767677	65.4112033271226\\
22.969696969697	65.6341679276176\\
23.2626262626263	65.8543070321182\\
23.5555555555556	66.071691357028\\
23.8484848484848	66.286388996698\\
24.1414141414141	66.4984655514744\\
24.4343434343434	66.7079842480241\\
24.7272727272727	66.9150060524899\\
25.020202020202	67.1195897769824\\
25.3131313131313	67.3217921798742\\
25.6060606060606	67.5216680603231\\
25.8989898989899	67.7192703474201\\
26.1919191919192	67.914650184323\\
26.4848484848485	68.1078570077116\\
26.7777777777778	68.2989386228727\\
27.0707070707071	68.4879412747005\\
27.3636363636364	68.6749097148757\\
27.6565656565657	68.8598872654678\\
27.9494949494949	69.042915879187\\
28.2424242424242	69.2240361964947\\
28.5353535353535	69.4032875997672\\
28.8282828282828	69.5807082646941\\
29.1212121212121	69.7563352090772\\
29.4141414141414	69.9302043391889\\
29.7070707070707	70.1023504938317\\
30	70.2728074862374\\
};
\addlegendentry{TRPL};

\addplot [color=mycolor4,solid]
  table[row sep=crcr]{%
1	30.7915254982822\\
1.29292929292929	33.9434222862678\\
1.58585858585859	35.5316616445846\\
1.87878787878788	36.4283000798175\\
2.17171717171717	37.1924057499839\\
2.46464646464646	37.8719108333566\\
2.75757575757576	38.5554435638167\\
3.05050505050505	39.2675570359689\\
3.34343434343434	40.0024398250219\\
3.63636363636364	40.7481405071393\\
3.92929292929293	41.4941128854543\\
4.22222222222222	42.2326695378301\\
4.51515151515152	42.95868364737\\
4.80808080808081	43.6689535858759\\
5.1010101010101	44.3616304081836\\
5.39393939393939	45.0357852429981\\
5.68686868686869	45.6911029557652\\
5.97979797979798	46.3276711042173\\
6.27272727272727	46.945836254252\\
6.56565656565657	47.5461066501664\\
6.85858585858586	48.1290864581166\\
7.15151515151515	48.6954314624322\\
7.44444444444444	49.2458193563392\\
7.73737373737374	49.7809299882535\\
8.03030303030303	50.3014324189399\\
8.32323232323232	50.8079766484679\\
8.61616161616162	51.3011885476373\\
8.90909090909091	51.7816669857065\\
9.2020202020202	52.2499824573702\\
9.49494949494949	52.7066767250179\\
9.78787878787879	53.152263139189\\
10.0808080808081	53.5872274020081\\
10.3737373737374	54.0120286094323\\
10.6666666666667	54.4271004579408\\
10.959595959596	54.8328525363514\\
11.2525252525253	55.2296716481953\\
11.5454545454545	55.6179231276041\\
11.8383838383838	55.9979521240634\\
12.1313131313131	56.3700848401753\\
12.4242424242424	56.7346297127695\\
12.7171717171717	57.0918785320623\\
13.010101010101	57.4421074965988\\
13.3030303030303	57.7855782038035\\
13.5959595959596	58.1225385773774\\
13.8888888888889	58.4532237337049\\
14.1818181818182	58.7778567900119\\
14.4747474747475	59.0966496173513\\
14.7676767676768	59.4098035416476\\
15.0606060606061	59.7175099960736\\
15.3535353535354	60.0199511279849\\
15.6464646464646	60.317300363539\\
15.9393939393939	60.6097229329916\\
16.2323232323232	60.8973763595041\\
16.5252525252525	61.1804109141314\\
16.8181818181818	61.4589700394879\\
17.1111111111111	61.7331907444194\\
17.4040404040404	62.0032039718441\\
17.6969696969697	62.2691349417683\\
17.989898989899	62.5311034713306\\
18.2828282828283	62.7892242735887\\
18.5757575757576	63.0436072366292\\
18.8686868686869	63.2943576844593\\
19.1616161616162	63.5415766210216\\
19.4545454545455	63.7853609585727\\
19.7474747474747	64.025803731563\\
20.040404040404	64.2629942970691\\
20.3333333333333	64.4970185227456\\
20.6262626262626	64.727958963188\\
20.9191919191919	64.9558950255269\\
21.2121212121212	65.1809031250124\\
21.5050505050505	65.4030568312845\\
21.7979797979798	65.622427005976\\
22.0909090909091	65.8390819322413\\
22.3838383838384	66.0530874367596\\
22.6767676767677	66.2645070047219\\
22.969696969697	66.4734018882686\\
23.2626262626263	66.679831208813\\
23.5555555555556	66.8838520536515\\
23.8484848484848	67.0855195672332\\
24.1414141414141	67.2848870374317\\
24.4343434343434	67.4820059771411\\
24.7272727272727	67.6769262014908\\
25.020202020202	67.8696959009536\\
25.3131313131313	68.0603617106055\\
25.6060606060606	68.2489687757714\\
25.8989898989899	68.4355608142797\\
26.1919191919192	68.620180175531\\
26.4848484848485	68.8028678965711\\
26.7777777777778	68.9836637553483\\
27.0707070707071	69.1626063213201\\
27.3636363636364	69.3397330035641\\
27.6565656565657	69.5150800965393\\
27.9494949494949	69.688682823631\\
28.2424242424242	69.8605753786071\\
28.5353535353535	70.0307909651024\\
28.8282828282828	70.1993618342431\\
29.1212121212121	70.366319320513\\
29.4141414141414	70.5316938759595\\
29.7070707070707	70.6955151028293\\
30	70.8578117847193\\
};
\addlegendentry{GWPL};

\addplot [color=mycolor5,solid]
  table[row sep=crcr]{%
1	49.3538609508286\\
1.29292929292929	53.3959745464176\\
1.58585858585859	56.5207156914694\\
1.87878787878788	59.0676402548208\\
2.17171717171717	61.2264933448044\\
2.46464646464646	63.1104587867558\\
2.75757575757576	64.7902302190781\\
3.05050505050505	66.311830849943\\
3.34343434343434	67.7064887792703\\
3.63636363636364	68.996278663249\\
3.92929292929293	70.1974189696354\\
4.22222222222222	71.3222456849008\\
4.51515151515152	72.3804266076039\\
4.80808080808081	73.3797314926537\\
5.1010101010101	74.3265363364305\\
5.39393939393939	75.2261642665338\\
5.68686868686869	76.0831231335442\\
5.97979797979798	76.9012758912276\\
6.27272727272727	77.6839659897684\\
6.56565656565657	78.4341118308212\\
6.85858585858586	79.1542793983295\\
7.15151515151515	79.8467391285975\\
7.44444444444444	80.5135111519851\\
7.73737373737374	81.1564017871898\\
8.03030303030303	81.7770333396581\\
8.32323232323232	82.3768686939538\\
8.61616161616162	82.9572318016748\\
8.90909090909091	83.5193248929627\\
9.2020202020202	84.0642430434446\\
9.49494949494949	84.5929865853301\\
9.78787878787879	85.1064717453882\\
10.0808080808081	85.6055398128794\\
10.3737373737374	86.0909650798918\\
10.6666666666667	86.5634617498139\\
10.959595959596	87.0236899732883\\
11.2525252525253	87.4722611423495\\
11.5454545454545	87.9097425507082\\
11.8383838383838	88.3366615099201\\
12.1313131313131	88.7535089964717\\
12.4242424242424	89.1607428928525\\
12.7171717171717	89.5587908758959\\
13.010101010101	89.9480529976065\\
13.3030303030303	90.3289039970187\\
13.5959595959596	90.7016953760679\\
13.8888888888889	91.0667572678091\\
14.1818181818182	91.4244001214087\\
14.4747474747475	91.7749162250337\\
14.7676767676768	92.1185810849709\\
15.0606060606061	92.4556546769273\\
15.3535353535354	92.7863825834364\\
15.6464646464646	93.1109970295529\\
15.9393939393939	93.4297178275263\\
16.2323232323232	93.7427532398573\\
16.5252525252525	94.0503007690251\\
16.8181818181818	94.3525478812105\\
17.1111111111111	94.6496726705026\\
17.4040404040404	94.9418444693439\\
17.6969696969697	95.2292244103331\\
17.989898989899	95.5119659439452\\
18.2828282828283	95.7902153162385\\
18.5757575757576	96.0641120101891\\
18.8686868686869	96.3337891539123\\
19.1616161616162	96.5993738986962\\
19.4545454545455	96.8609877694759\\
19.7474747474747	97.1187469901163\\
20.040404040404	97.3727627856349\\
20.3333333333333	97.6231416632925\\
20.6262626262626	97.8699856742928\\
20.9191919191919	98.1133926576673\\
21.2121212121212	98.3534564677773\\
21.5050505050505	98.5902671867284\\
21.7979797979798	98.8239113228803\\
22.0909090909091	99.0544719965241\\
22.3838383838384	99.2820291137069\\
22.6767676767677	99.5066595290969\\
22.969696969697	99.7284371987036\\
23.2626262626263	99.9474333232019\\
23.5555555555556	100.163716482539\\
23.8484848484848	100.377352762456\\
24.1414141414141	100.588405873489\\
24.4343434343434	100.796937262992\\
24.7272727272727	101.003006220645\\
25.020202020202	101.206669977924\\
25.3131313131313	101.407983801904\\
25.6060606060606	101.607001083818\\
25.8989898989899	101.80377342268\\
26.1919191919192	101.99835070433\\
26.4848484848485	102.190781176174\\
26.7777777777778	102.381111517909\\
27.0707070707071	102.569386908486\\
27.3636363636364	102.755651089543\\
27.6565656565657	102.939946425537\\
27.9494949494949	103.12231396077\\
28.2424242424242	103.302793473501\\
28.5353535353535	103.48142352733\\
28.8282828282828	103.658241519998\\
29.1212121212121	103.833283729774\\
29.4141414141414	104.006585359566\\
29.7070707070707	104.178180578881\\
30	104.348102563768\\
};
\addlegendentry{NSPL};

\addplot [color=black,solid]
  table[row sep=crcr]{%
1	11.1866992946941\\
2	23.2278991212534\\
4	35.2690989478126\\
8	47.3102987743719\\
15	58.2303496569214\\
30	70.2715494834806\\
};
\addlegendentry{Proposed model};
\end{axis}
\end{tikzpicture}%
\caption{Comparison between proposed model, measured path loss existing PL models at a height of 0.14 m and 2.02 m for Rx and TX respectively.}
\label{ourModel2}
\end{figure}

\begin{figure}[H]
\centering
% This file was created by matlab2tikz.
%
%The latest updates can be retrieved from
%  http://www.mathworks.com/matlabcentral/fileexchange/22022-matlab2tikz-matlab2tikz
%where you can also make suggestions and rate matlab2tikz.
%
\definecolor{mycolor1}{rgb}{0.00000,0.44700,0.74100}%
\definecolor{mycolor2}{rgb}{0.85000,0.32500,0.09800}%
\definecolor{mycolor3}{rgb}{0.92900,0.69400,0.12500}%
\definecolor{mycolor4}{rgb}{0.49400,0.18400,0.55600}%
\definecolor{mycolor5}{rgb}{0.46600,0.67400,0.18800}%
%
\begin{tikzpicture}

\begin{axis}[%
width=2.8in,
height=\myvar,
at={(0.758in,0.481in)},
scale only axis,
xmode=log,
extra x ticks={2,5,20}, 
extra x tick style={log identify minor tick positions=false},
log ticks with fixed point,
xmin=0.9,
xmax=31,
xlabel=Distance (m),
xminorticks=true,
xmajorgrids,
xminorgrids,
ymin=20,
ymax=100,
ylabel=Path loss (dB),
ymajorgrids,
axis background/.style={fill=white},
legend style={legend cell align=left,align=left,draw=white!15!black},
legend pos = north west
]
\addplot [color=mycolor1,mark size=2.5pt,only marks,mark=asterisk,mark options={solid}]
  table[row sep=crcr]{%
1.03029316216308	31.28383923555\\
2.01531734473755	39.0242931377669\\
4.00768062599803	49.7368915447371\\
8.00384307692249	59.2686962142928\\
15.0020499932509	68.2995914203388\\
30.0010250491546	72.5078313840168\\
};
\addlegendentry{Measured PL};

\addplot [color=mycolor2,solid]
  table[row sep=crcr]{%
1	31.1115179430228\\
1.29292929292929	33.3430134440292\\
1.58585858585859	35.1168070992565\\
1.87878787878788	36.5890729354301\\
};
\addlegendentry{FSPL};
\addplot [color=mycolor2,dashed, forget plot]
  table[row sep=crcr]{%
1.87878787878788	36.5890729354301\\
2.17171717171717	37.8475832493839\\
2.46464646464646	38.9466105778464\\
2.75757575757576	39.9220669918869\\
3.05050505050505	40.7989529102148\\
3.34343434343434	41.5953739265862\\
3.63636363636364	42.3248640664175\\
3.92929292929293	42.9978060775859\\
4.22222222222222	43.6223396865725\\
4.51515151515152	44.2049645137105\\
4.80808080808081	44.7509531054816\\
5.1010101010101	45.264641613445\\
5.39393939393939	45.7496391916429\\
5.68686868686869	46.2089819480987\\
5.97979797979798	46.6452481855302\\
6.27272727272727	47.0606460546034\\
6.56565656565657	47.4570811839289\\
6.85858585858586	47.8362095366818\\
7.15151515151515	48.1994792048672\\
7.44444444444444	48.5481638082528\\
7.73737373737374	48.8833894437239\\
8.03030303030303	49.2061566242012\\
8.32323232323232	49.5173582850141\\
8.61616161616162	49.8177946744222\\
8.90909090909091	50.1081857537082\\
9.2020202020202	50.3891815905317\\
9.49494949494949	50.6613711230658\\
9.78787878787879	50.9252895920871\\
10.0808080808081	51.1814248768192\\
10.3737373737374	51.4302229230174\\
10.6666666666667	51.6720924150277\\
10.959595959596	51.9074088147627\\
11.2525252525253	52.136517867826\\
11.5454545454545	52.3597386589774\\
11.8383838383838	52.5773662847132\\
12.1313131313131	52.7896741991299\\
12.4242424242424	52.9969162798597\\
12.7171717171717	53.199328653229\\
13.010101010101	53.3971313115477\\
13.3030303030303	53.5905295503075\\
13.5959595959596	53.7797152488309\\
13.8888888888889	53.9648680143974\\
14.1818181818182	54.1461562069475\\
14.4747474747475	54.3237378590187\\
14.7676767676768	54.4977615035086\\
15.0606060606061	54.6683669201317\\
15.3535353535354	54.8356858099672\\
15.6464646464646	54.9998424062559\\
15.9393939393939	55.1609540285398\\
16.2323232323232	55.3191315863387\\
16.5252525252525	55.4744800377779\\
16.8181818181818	55.6270988079186\\
17.1111111111111	55.7770821709656\\
17.4040404040404	55.9245196000324\\
17.6969696969697	56.069496087713\\
17.989898989899	56.2120924403367\\
18.2828282828283	56.3523855484555\\
18.5757575757576	56.4904486358333\\
18.8686868686869	56.6263514889533\\
19.1616161616162	56.760160668845\\
19.4545454545455	56.8919397068421\\
19.7474747474747	57.0217492857095\\
20.040404040404	57.149647407435\\
20.3333333333333	57.2756895488449\\
20.6262626262626	57.3999288060896\\
20.9191919191919	57.5224160289408\\
21.2121212121212	57.6431999457502\\
21.5050505050505	57.7623272798382\\
21.7979797979798	57.8798428580096\\
22.0909090909091	57.9957897118245\\
22.3838383838384	58.1102091721996\\
22.6767676767677	58.2231409578586\\
22.969696969697	58.3346232581061\\
23.2626262626263	58.4446928103564\\
23.5555555555556	58.5533849728113\\
23.8484848484848	58.6607337926463\\
24.1414141414141	58.7667720700345\\
24.4343434343434	58.8715314183094\\
24.7272727272727	58.9750423205423\\
25.020202020202	59.0773341827885\\
25.3131313131313	59.1784353842344\\
25.6060606060606	59.2783733244589\\
25.8989898989899	59.3771744680074\\
26.1919191919192	59.4748643864588\\
26.4848484848485	59.5714677981531\\
26.7777777777778	59.6670086057337\\
27.0707070707071	59.7615099316476\\
27.3636363636364	59.8549941517351\\
27.6565656565657	59.9474829270312\\
27.9494949494949	60.0389972338908\\
28.2424242424242	60.1295573925447\\
28.5353535353535	60.2191830941809\\
28.8282828282828	60.3078934266443\\
29.1212121212121	60.3957068988359\\
29.4141414141414	60.4826414638918\\
29.7070707070707	60.5687145412132\\
30	60.653943037416\\
};

\addplot [color=mycolor3,dashed, forget plot]
  table[row sep=crcr]{%
1	25.7293491507274\\
1.29292929292929	30.1923401527402\\
1.58585858585859	33.7399274631948\\
1.87878787878788	36.6844591355421\\
};
\addplot [color=mycolor3,solid]
  table[row sep=crcr]{%
1.87878787878788	36.6844591355421\\
2.17171717171717	39.2014797634496\\
2.46464646464646	41.3995344203746\\
2.75757575757576	43.3504472484557\\
3.05050505050505	45.1042190851115\\
3.34343434343434	46.6970611178542\\
3.63636363636364	48.1560413975169\\
3.92929292929293	49.5019254198537\\
4.22222222222222	50.7509926378268\\
4.51515151515152	51.9162422921029\\
4.80808080808081	53.0082194756451\\
5.1010101010101	54.0355964915719\\
5.39393939393939	55.0055916479677\\
5.68686868686869	55.9242771608793\\
5.97979797979798	56.7968096357422\\
6.27272727272727	57.6276053738886\\
6.56565656565657	58.4204756325396\\
6.85858585858586	59.1787323380455\\
7.15151515151515	59.9052716744162\\
7.44444444444444	60.6026408811875\\
7.73737373737374	61.2730921521296\\
8.03030303030303	61.9186265130842\\
8.32323232323232	62.5410298347101\\
8.61616161616162	63.1419026135263\\
8.90909090909091	63.7226847720982\\
9.2020202020202	64.2846764457454\\
9.49494949494949	64.8290555108134\\
9.78787878787879	65.356892448856\\
10.0808080808081	65.8691630183203\\
10.3737373737374	66.3667591107166\\
10.6666666666667	66.8504980947372\\
10.959595959596	67.3211308942074\\
11.2525252525253	67.7793490003338\\
11.5454545454545	68.2257905826367\\
11.8383838383838	68.6610458341083\\
12.1313131313131	69.0856616629417\\
12.4242424242424	69.5001458244013\\
12.7171717171717	69.9049705711399\\
13.010101010101	70.3005758877772\\
13.3030303030303	70.6873723652968\\
13.5959595959596	71.0657437623437\\
13.8888888888889	71.4360492934767\\
14.1818181818182	71.7986256785769\\
14.4747474747475	72.1537889827192\\
14.7676767676768	72.5018362716991\\
15.0606060606061	72.8430471049452\\
15.3535353535354	73.1776848846163\\
15.6464646464646	73.5059980771937\\
15.9393939393939	73.8282213217615\\
16.2323232323232	74.1445764373592\\
16.5252525252525	74.4552733402376\\
16.8181818181818	74.760510880519\\
17.1111111111111	75.0604776066129\\
17.4040404040404	75.3553524647466\\
17.6969696969697	75.6453054401079\\
17.989898989899	75.9304981453552\\
18.2828282828283	76.2110843615928\\
18.5757575757576	76.4872105363485\\
18.8686868686869	76.7590162425884\\
19.1616161616162	77.0266346023719\\
19.4545454545455	77.2901926783661\\
19.7474747474747	77.5498118361009\\
20.040404040404	77.8056080795518\\
20.3333333333333	78.0576923623716\\
20.6262626262626	78.3061708768611\\
20.9191919191919	78.5511453225635\\
21.2121212121212	78.7927131561822\\
21.5050505050505	79.0309678243583\\
21.7979797979798	79.2659989807011\\
22.0909090909091	79.4978926883309\\
22.3838383838384	79.7267316090811\\
22.6767676767677	79.9525951803991\\
22.969696969697	80.1755597808941\\
23.2626262626263	80.3956988853947\\
23.5555555555556	80.6130832103045\\
23.8484848484848	80.8277808499745\\
24.1414141414141	81.0398574047509\\
24.4343434343434	81.2493761013006\\
24.7272727272727	81.4563979057664\\
25.020202020202	81.6609816302589\\
25.3131313131313	81.8631840331507\\
25.6060606060606	82.0630599135996\\
25.8989898989899	82.2606622006966\\
26.1919191919192	82.4560420375995\\
26.4848484848485	82.649248860988\\
26.7777777777778	82.8403304761491\\
27.0707070707071	83.029333127977\\
27.3636363636364	83.2163015681522\\
27.6565656565657	83.4012791187443\\
27.9494949494949	83.5843077324635\\
28.2424242424242	83.7654280497712\\
28.5353535353535	83.9446794530437\\
28.8282828282828	84.1221001179706\\
29.1212121212121	84.2977270623537\\
29.4141414141414	84.4715961924654\\
29.7070707070707	84.6437423471082\\
30	84.8141993395139\\
};
\addlegendentry{ATRPL};

\addplot [color=mycolor4,solid]
  table[row sep=crcr]{%
1	31.2328670540324\\
1.29292929292929	34.5552771726006\\
1.58585858585859	37.3693273923224\\
1.87878787878788	39.8021175378882\\
2.17171717171717	41.9417732796322\\
2.46464646464646	43.8500753195524\\
2.75757575757576	45.5715326004799\\
3.05050505050505	47.1391140986627\\
3.34343434343434	48.5778586915229\\
3.63636363636364	49.907201725863\\
3.92929292929293	51.1425171498245\\
4.22222222222222	52.2961680346275\\
4.51515151515152	53.3782404075676\\
4.80808080808081	54.3970675919587\\
5.1010101010101	55.3596125255927\\
5.39393939393939	56.2717516277483\\
5.68686868686869	57.1384890271152\\
5.97979797979798	57.9641206236034\\
6.27272727272727	58.7523614092353\\
6.56565656565657	59.5064454731815\\
6.85858585858586	60.2292054182088\\
7.15151515151515	60.9231360635282\\
7.44444444444444	61.5904460162021\\
7.73737373737374	62.2330997771672\\
8.03030303030303	62.8528523896516\\
8.32323232323232	63.4512781586071\\
8.61616161616162	64.0297946168024\\
8.90909090909091	64.5896826502949\\
9.2020202020202	65.1321034981107\\
9.49494949494949	65.6581131905776\\
9.78787878787879	66.1686748754342\\
10.0808080808081	66.6646693916562\\
10.3737373737374	67.1469043814165\\
10.6666666666667	67.6161221759992\\
10.959595959596	68.0730066482984\\
11.2525252525253	68.5181891901524\\
11.5454545454545	68.952253945211\\
11.8383838383838	69.3757424058332\\
12.1313131313131	69.7891574645089\\
12.4242424242424	70.1929669956373\\
12.7171717171717	70.587607031475\\
13.010101010101	70.9734845861851\\
13.3030303030303	71.3509801737386\\
13.5959595959596	71.7204500586334\\
13.8888888888889	72.0822282727288\\
14.1818181818182	72.4366284267585\\
14.4747474747475	72.7839453410909\\
14.7676767676768	73.124456516952\\
15.0606060606061	73.4584234664701\\
15.3535353535354	73.7860929174908\\
15.6464646464646	74.107697907043\\
15.9393939393939	74.4234587755782\\
16.2323232323232	74.7335840725919\\
16.5252525252525	75.0382713829375\\
16.8181818181818	75.3377080820199\\
17.1111111111111	75.6320720270891\\
17.4040404040404	75.9215321910119\\
17.6969696969697	76.2062492441702\\
17.989898989899	76.4863760894967\\
18.2828282828283	76.7620583551055\\
18.5757575757576	77.0334348484849\\
18.8686868686869	77.3006379757973\\
19.1616161616162	77.5637941294522\\
19.4545454545455	77.8230240467894\\
19.7474747474747	78.0784431424199\\
20.040404040404	78.3301618165095\\
20.3333333333333	78.5782857410666\\
20.6262626262626	78.8229161260898\\
20.9191919191919	79.0641499672516\\
21.2121212121212	79.3020802766341\\
21.5050505050505	79.5367962978896\\
21.7979797979798	79.7683837070705\\
22.0909090909091	79.9969248002582\\
22.3838383838384	80.2224986690195\\
22.6767676767677	80.4451813646256\\
22.969696969697	80.6650460518852\\
23.2626262626263	80.8821631533718\\
23.5555555555556	81.0966004847547\\
23.8484848484848	81.3084233818847\\
24.1414141414141	81.5176948202294\\
24.4343434343434	81.7244755272047\\
24.7272727272727	81.9288240879035\\
25.020202020202	82.1307970446811\\
25.3131313131313	82.3304489910221\\
25.6060606060606	82.5278326600761\\
25.8989898989899	82.7229990082247\\
26.1919191919192	82.9159972940064\\
26.4848484848485	83.1068751527101\\
26.7777777777778	83.2956786669153\\
27.0707070707071	83.4824524332427\\
27.3636363636364	83.6672396255565\\
27.6565656565657	83.8500820548418\\
27.9494949494949	84.0310202259653\\
28.2424242424242	84.2100933915115\\
28.5353535353535	84.3873396028744\\
28.8282828282828	84.5627957587687\\
29.1212121212121	84.7364976513186\\
29.4141414141414	84.9084800098642\\
29.7070707070707	85.0787765426229\\
30	85.2474199763289\\
};
\addlegendentry{GWPL};

\addplot [color=mycolor5,dashed, forget plot]
  table[row sep=crcr]{%
1	46.255884046764\\
1.29292929292929	50.3667792108431\\
1.58585858585859	53.7129556038272\\
1.87878787878788	56.5334445774278\\
2.17171717171717	58.9693259240209\\
2.46464646464646	61.1116669568395\\
2.75757575757576	63.0227888618319\\
3.05050505050505	64.7472078890716\\
3.34343434343434	66.3178061197953\\
3.63636363636364	67.7595427015522\\
3.92929292929293	69.0917976531737\\
4.22222222222222	70.3299112717641\\
4.51515151515152	71.4862286697426\\
4.80808080808081	72.5708284004176\\
5.1010101010101	73.5920429986904\\
5.39393939393939	74.5568387058472\\
5.68686868686869	75.4710976217367\\
5.97979797979798	76.3398308207934\\
6.27272727272727	77.1673417044993\\
6.56565656565657	77.9573528781169\\
6.85858585858586	78.7131058849294\\
7.15151515151515	79.4374404645581\\
7.44444444444444	80.1328581704067\\
7.73737373737374	80.8015739021597\\
8.03030303030303	81.445558002174\\
8.32323232323232	82.0665709122563\\
8.61616161616162	82.6661919120683\\
8.90909090909091	83.2458431100063\\
9.2020202020202	83.8068095961798\\
9.49494949494949	84.3502564703531\\
9.78787878787879	84.8772433080773\\
10.0808080808081	85.3887365134106\\
10.3737373737374	85.8856199177647\\
10.6666666666667	86.3687039151034\\
10.959595959596	86.8387333692543\\
11.2525252525253	87.2963944859915\\
11.5454545454545	87.7423208082174\\
11.8383838383838	88.1770984650457\\
12.1313131313131	88.6012707834017\\
12.4242424242424	89.0153423527584\\
12.7171717171717	89.4197826189594\\
13.010101010101	89.8150290710632\\
13.3030303030303	90.2014900752454\\
13.5959595959596	90.5795474016135\\
13.8888888888889	90.9495584829919\\
14.1818181818182	91.3118584390615\\
14.4747474747475	91.6667618944895\\
14.7676767676768	92.0145646156927\\
15.0606060606061	92.3555449875062\\
15.3535353535354	92.6899653481769\\
15.6464646464646	93.0180731986774\\
15.9393939393939	93.3401023002674\\
16.2323232323232	93.6562736724646\\
16.5252525252525	93.9667965020686\\
16.8181818181818	94.2718689725823\\
17.1111111111111	94.5716790222449\\
17.4040404040404	94.8664050379237\\
17.6969696969697	95.156216491265\\
17.989898989899	95.4412745227738\\
18.2828282828283	95.7217324788515\\
18.5757575757576	95.9977364062655\\
18.8686868686869	96.2694255080334\\
19.1616161616162	96.5369325642792\\
19.4545454545455	96.800384321241\\
19.7474747474747	97.0599018512771\\
20.040404040404	97.3156008864277\\
20.3333333333333	97.5675921278272\\
20.6262626262626	97.8159815330356\\
20.9191919191919	98.0608705831519\\
21.2121212121212	98.3023565313931\\
21.5050505050505	98.5405326346597\\
21.7979797979798	98.7754883694668\\
22.0909090909091	99.0073096334894\\
22.3838383838384	99.2360789338566\\
22.6767676767677	99.4618755632268\\
22.969696969697	99.6847757645828\\
23.2626262626263	99.9048528856026\\
23.5555555555556	100.122177523387\\
23.8484848484848	100.336817660258\\
24.1414141414141	100.548838791281\\
24.4343434343434	100.758304044108\\
24.7272727272727	100.965274291692\\
25.020202020202	101.169808258371\\
25.3131313131313	101.37196261979\\
25.6060606060606	101.571792097078\\
25.8989898989899	101.769349545682\\
26.1919191919192	101.964686039207\\
26.4848484848485	102.157850948603\\
26.7777777777778	102.348892016996\\
27.0707070707071	102.537855430464\\
27.3636363636364	102.724785885\\
27.6565656565657	102.909726649917\\
27.9494949494949	103.092719627916\\
28.2424242424242	103.273805412027\\
28.5353535353535	103.453023339616\\
28.8282828282828	103.630411543635\\
29.1212121212121	103.806007001286\\
29.4141414141414	103.97984558025\\
29.7070707070707	104.151962082632\\
30	104.322390286748\\
};
\addplot [color=mycolor5,solid]
  table[row sep=crcr]{%
  30	104.322390286748\\
  };
\addlegendentry{NSPL};


\addplot [color=black,dashed, forget plot]
  table[row sep=crcr]{%
1	25.7293491507274\\
1.29292929292929	30.1923401527402\\
1.58585858585859	33.7399274631948\\
1.87878787878788	36.6844591355421\\
};
\addplot [color=black,solid]
  table[row sep=crcr]{%
1.87878787878788	36.6844591355421\\
2.17171717171717	39.2014797634496\\
2.46464646464646	41.3995344203746\\
2.75757575757576	43.3504472484557\\
3.05050505050505	45.1042190851115\\
3.34343434343434	46.6970611178542\\
3.63636363636364	48.1560413975169\\
3.92929292929293	49.5019254198537\\
4.22222222222222	50.7509926378268\\
4.51515151515152	51.9162422921029\\
4.80808080808081	53.0082194756451\\
5.1010101010101	54.0355964915719\\
5.39393939393939	55.0055916479677\\
5.68686868686869	55.9242771608793\\
5.97979797979798	56.7968096357422\\
6.27272727272727	57.6276053738886\\
6.56565656565657	58.4204756325396\\
6.85858585858586	59.1787323380455\\
7.15151515151515	59.9052716744162\\
7.44444444444444	60.6026408811875\\
7.73737373737374	61.2730921521296\\
8.03030303030303	61.9186265130842\\
8.32323232323232	62.5410298347101\\
8.61616161616162	63.1419026135263\\
8.90909090909091	63.7226847720982\\
9.2020202020202	64.2846764457454\\
9.49494949494949	64.8290555108134\\
9.78787878787879	65.356892448856\\
10.0808080808081	65.8691630183203\\
10.3737373737374	66.3667591107166\\
10.6666666666667	66.8504980947372\\
10.959595959596	67.3211308942074\\
11.2525252525253	67.7793490003338\\
11.5454545454545	68.2257905826367\\
11.8383838383838	68.6610458341083\\
12.1313131313131	69.0856616629417\\
12.4242424242424	69.5001458244013\\
12.7171717171717	69.9049705711399\\
13.010101010101	70.3005758877772\\
13.3030303030303	70.6873723652968\\
13.5959595959596	71.0657437623437\\
13.8888888888889	71.4360492934767\\
14.1818181818182	71.7986256785769\\
14.4747474747475	72.1537889827192\\
14.7676767676768	72.5018362716991\\
15.0606060606061	72.8430471049452\\
15.3535353535354	73.1776848846163\\
15.6464646464646	73.5059980771937\\
15.9393939393939	73.8282213217615\\
16.2323232323232	74.1445764373592\\
16.5252525252525	74.4552733402376\\
16.8181818181818	74.760510880519\\
17.1111111111111	75.0604776066129\\
17.4040404040404	75.3553524647466\\
17.6969696969697	75.6453054401079\\
17.989898989899	75.9304981453552\\
18.2828282828283	76.2110843615928\\
18.5757575757576	76.4872105363485\\
18.8686868686869	76.7590162425884\\
19.1616161616162	77.0266346023719\\
19.4545454545455	77.2901926783661\\
19.7474747474747	77.5498118361009\\
20.040404040404	77.8056080795518\\
20.3333333333333	78.0576923623716\\
20.6262626262626	78.3061708768611\\
20.9191919191919	78.5511453225635\\
21.2121212121212	78.7927131561822\\
21.5050505050505	79.0309678243583\\
21.7979797979798	79.2659989807011\\
22.0909090909091	79.4978926883309\\
22.3838383838384	79.7267316090811\\
22.6767676767677	79.9525951803991\\
22.969696969697	80.1755597808941\\
23.2626262626263	80.3956988853947\\
23.5555555555556	80.6130832103045\\
23.8484848484848	80.8277808499745\\
24.1414141414141	81.0398574047509\\
24.4343434343434	81.2493761013006\\
24.7272727272727	81.4563979057664\\
25.020202020202	81.6609816302589\\
25.3131313131313	81.8631840331507\\
25.6060606060606	82.0630599135996\\
25.8989898989899	82.2606622006966\\
26.1919191919192	82.4560420375995\\
26.4848484848485	82.649248860988\\
26.7777777777778	82.8403304761491\\
27.0707070707071	83.029333127977\\
27.3636363636364	83.2163015681522\\
27.6565656565657	83.4012791187443\\
27.9494949494949	83.5843077324635\\
28.2424242424242	83.7654280497712\\
28.5353535353535	83.9446794530437\\
28.8282828282828	84.1221001179706\\
29.1212121212121	84.2977270623537\\
29.4141414141414	84.4715961924654\\
29.7070707070707	84.6437423471082\\
30	84.8141993395139\\
};
\addlegendentry{Proposed Model};

\end{axis}
\end{tikzpicture}%
\caption{Comparison between measured PL and predicted PL from models at a height of 0.14 m and 0.36 m for the Tx and Rx respectively.}
\label{Models62}
\end{figure}


The accuracy is found on the same data set as the other models as none of the data points have been used for the derivation of any part of the model. The MSE is found to 87.66, which is significantly lower than either ATRPL or NSPL. It can therefore be concluded that the weakness of the ATRPL at very low Tx and/or Rx heights has been strengthened. The same can be said for the NSPL where the applicability has been greatly extended, the performance at low heights is still affected by the magnitude of $z$ as seen in \autoref{ourModel1}. However, as the heights increases the dependency of $z$ lessen, as the contribution from the NSPL part becomes smaller. This can be seen as the model is almost equal to the ATRPL on \autoref{ourModel2} and \autoref{Models62}.  The model does still not perform as well as the GWPL, however that could not be expected, as the proposed model still uses approximations and simplifications. When comparing the proposed PL model with the GWPL, the part where the proposed PL model, has an edge over the GWPL, is in the determination of $z$. For the proposed PL model, only the magnitude of $z$ is used, neglecting the complex part of $z$, meaning that the phase is not an issue. As for the GWPL $z$ is also a part of $R$ see \eqref{reflection_coefficient} and $A$ see \eqref{attenuation_factor_A}, in where the magnitude of $z$ is not taken, as the complex part also has an influence. However, with a better estimate of $z$ both the proposed PL model and the GWPL might fit better at low Tx and Rx heights which can be seen on \autoref{Modelzadjust} in the case of the proposed PL model. 


