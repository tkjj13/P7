\subsection{Model estimation}

As it is seen from the figures \autoref{Models1}, \autoref{Models7}, \autoref{Models10}, the different models, besides GWPL, have different areas where they are accurate, will having other areas, where they either over and underestimate the PL. To make a fair comparison of their accuracy, there will taken a mean square error (MSE) over the areas, where the conditions for the different models is for filled.




%Conditions
%GWPL = none
%TRPL = (virker bedst ved højde over lambda)
%TRPL approx = d > dc = (4 pi ht hr)/lambda
%NSPL = h < lambda
%FSPL = no multipath



\begin{table}[!htbp]
\centering
\begin{tabular}{|l|l|l|}
\hline
\textbf{Models} & \textbf{MSE} & \textbf{Covarge in \%} \\ \hline
FSPL            & 15.95        & 35 \%                  \\ \hline
TRPL            & 7.73         & 30 \%                  \\ \hline
TRPL approx.    & 9.44         & 6.7 \%                 \\ \hline
GWPL            & 35.49        & 100 \%                 \\ \hline
NSPL            & 230.05       & 30 \%                  \\ \hline
\end{tabular}
\caption{The MSE of the different models, inside the areas, where the different models is accurate. The coverage, is the percent of the measurement points, that is inside this area.}
\label{my-label}
\end{table}






