\subsection{Important points}
The ideal PL model is the GWPL model \eqref{ground_wave} as it can be seen from the  \autoref{my-label}, the GWPL model gets the best coverage of $100\%$ ,as it takes all waves into account, and has a MSE of 35.49. But is also the most complex, as it requires to make measurements at desired locations, and requires rather complex data processing. The complexity of the GWPL model, is mostly due to the Complex relative permittivity $\epsilon_{0}$, as it requires to make measurements. 
\\
\\
It is also important to note that the three approximated models of the GWPL \eqref{simple_friss}\eqref{two_ray_model}\eqref{surface_wave}, which are based on the GWPL model are PL models that can be applied, and will have the same effect as the the GWPL model, and are more simple to apply. Although the surface wave model is based on the parameter $z$, which requires the same measurements as $\epsilon_{0}$, for this a average value of $z$ will be made, which is valid for the two surfaces the measurements have been made, as $z$ is different from surface to surface. It shall also be noted from \autoref{my-label} that all PL models have their coverage area and give a different MSE, given the conditions of the PL models. It shall also be noted that the larger the area the PL model covers, the larger the MSE will be, and as some of the measured points, have a larger dive, it also results in a larger MSE, as the measured points are not on a straight line. 



