\subsection*{Important points}
The most optimal PL model is the GWPL model \eqref{ground_wave} as it can be seen from the  \autoref{model_comparison}, the GWPL model gets the best coverage of $100\%$, as it takes all waves into account, and has a MSE of 35.49. But is also the most complex, as it requires to make measurements at desired locations. Further more by looking at \autoref{model_comparison} it can be seen that by making the simplification of the GWPL, not only loses the other models applicability, as it is now necessary to calculate the conditions before assessing the PL, but also the inaccuracy of the prediction increases. 
This is partly because some of the measured points deviate from the tendencies due to unknown factors, also from \autoref{model_comparison} can it be seen that the MSE for the NSPL is quite large, this is believed to primarily be due to measurement uncertainty of $\epsilon_{0}$.  
But the main reason of the loss of prediction accuracy is due to some weak points of the simplifications made. It can be seen that FSPL actually predicts quite well inside its valid region, but it is also the model that has second fewest valid points in the measurement campaign, due to the assumption of no multipath. Where TRPL has the highest valid region of the simplified models has a rather high MSE, this is due to its lack of ability to account for the surface wave, as seen from \eqref{two_ray_model} when the heights go to zero so does the power received. The NSPL is worst off, lacking in both applicability as well as accuracy, the weak point here is primarily the need of $z$. This leaves all model with crucial weak points. This paper will now propose a model that accounts for some of these weak points. 

%It is also important to note that the three approximated models of the GWPL \eqref{simple_friss}\eqref{two_ray_model}\eqref{surface_wave}, uses some restriction that make it possible to simplify the equation, but still account for the dominant factors of the GWPL model. Although the NSPL model is based on the parameter $z$, which requires the same measurements as $\epsilon_{0}$, only the size of $z$ is necessary and it can therefore be rather easily estimated. 




