\subsection*{Important points}
The most optimal PL model is the GWPL model \eqref{ground_wave} as it can be seen from the  \autoref{model_comparison}, the GWPL model gets the best coverage of $100\%$, as it takes all waves into account, and has a MSE of 35.49. But is also the most complex, as it requires to make measurements at desired locations, and requires rather complex data processing. The complexity of the GWPL model, is mostly due to the Complex relative permittivity $\epsilon_{0}$, as it requires to make measurements. 
\\
\\
It is also important to note that the three approximated models of the GWPL \eqref{simple_friss}\eqref{two_ray_model}\eqref{surface_wave}, uses some restriction that make it possible to simplify the equation, but still account for the dominant factors of the GWPL model. Although the NSPL model is based on the parameter $z$, which requires the same measurements as $\epsilon_{0}$, only the size of $z$ is necessary and it can therefore be rather easily estimated. It shall be noted that the larger the area the PL model covers, the larger the MSE might be, as some of the measured points deviate from the tendencies due to unknown factors. It can also be seen from the \autoref{model_comparison}, that the MSE for the NSPL is quite large, this is believed to be due to measurement uncertainty of $\epsilon_{0}$.  



