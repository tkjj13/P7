\vspace{3em}
The purpose of this paper is to get a better understanding of the path loss (PL) in near-ground scenarios, as wireless sensor networks will be more common in the future and this can require that placement of the antennas to be near ground, which can complicate the calculations of the PL. 

An investigation is made of four different PL models to explore the accuracy and applicability of the PL models, including weak points and strong points. The PL models considered are: ground wave (GWPL), Friss free space (FSPL), approximated two-ray ground reflection (ATRPL) and the Norton surface wave (NSPL). Supporting this investigation is a series of measurements conducted with six parameters: location, antenna type, polarization, height of transmitter, height of receiver, distance between transmitter and receiver. The frequency used for the measurements are 858 MHz. 

The measurements indicates that the polarization, antenna type and environment, have a less influence on the PL compared to the distance and the heights of the antennas. The results validate the conditions of the PL models coverage areas, as different tendencies are seen, in the different areas, that follows the PL models. The GWPL fits best to the results, even with the biggest coverage area, but has the disadvantage of being complicated and needs a complex surface constant, $z$. When getting close to the ground the result is depended $z$, where a wrong measurement of the $z$ can give a big offset, which influences both the GWPL and NSPL. 
 
Lastly, a new PL model is constructed which is based on the ATRPL and the NSPL. It is still subject to the condition of the ATRPL model resulting in the model’s coverage area to be the same as ATRPL. Improvement of ATRPL’s weak point, at near-ground placement of the antennas, is achieved by the NSPL at the cost of the introduction of the surface constant used to find the surface wave. The prediction accuracy of the proposed model exceeds the individual PL models, while still covering the same area as the ATRPL. Making it the best alternative to GWPL. The gain in using the proposed model lies in its simplicity computational wise. Another advantage is that the GWPL needs the complex value of $z$ where only the magnitude of $z$ is needed for the proposed model making it easier to estimate.