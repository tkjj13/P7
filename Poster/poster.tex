% --------------------------------------------------------------------------- %
% Poster for the ECCS 2011 Conference about Elementary Dynamic Networks.      %
% --------------------------------------------------------------------------- %
% Created with Brian Amberg's LaTeX Poster Template. Please refer for the     %
% attached README.md file for the details how to compile with `pdflatex`.     %
% --------------------------------------------------------------------------- %
% $LastChangedDate:: 2011-09-11 10:57:12 +0200 (V, 11 szept. 2011)          $ %
% $LastChangedRevision:: 128                                                $ %
% $LastChangedBy:: rlegendi                                                 $ %
% $Id:: poster.tex 128 2011-09-11 08:57:12Z rlegendi                        $ %
% --------------------------------------------------------------------------- %
\documentclass[a0paper,landscape]{baposter}
\usepackage[utf8]{inputenc}
\usepackage{relsize}		% For \smaller
\usepackage{url}			% For \url
\usepackage{epstopdf}	% Included EPS files automatically converted to PDF to include with pdflatex
\usepackage{amsmath,amssymb}
\usepackage{tikz}
\usepackage[americanresistors,americaninductors,americancurrents, americanvoltages]{circuitikz}

\usepackage{pdfpages}
\usepackage{pgfplots}
\pgfplotsset{filter discard warning = true, unbounded coords=discard}
\pgfplotsset{compat = newest}

%%% Global Settings %%%%%%%%%%%%%%%%%%%%%%%%%%%%%%%%%%%%%%%%%%%%%%%%%%%%%%%%%%%

\graphicspath{{pix/}}	% Root directory of the pictures 
\tracingstats=2			% Enabled LaTeX logging with conditionals
\usepackage[font=large,labelfont=bf]{caption}
%%% Color Definitions %%%%%%%%%%%%%%%%%%%%%%%%%%%%%%%%%%%%%%%%%%%%%%%%%%%%%%%%%

\definecolor{bordercol}{RGB}{40,40,40}
\definecolor{headercol1}{RGB}{186,215,230}
\definecolor{headercol2}{RGB}{80,80,80}
\definecolor{headerfontcol}{RGB}{0,0,0}
\definecolor{boxcolor}{RGB}{230,245,254}
\definecolor{thomasred}{RGB}{247,73,15} %217
\definecolor{thomasblue}{RGB}{0,114,189}
\definecolor{thomasyellow}{RGB}{237,179,32}
\definecolor{thomaspurple}{RGB}{126,47,142}
\definecolor{thomasgreen}{RGB}{119,172,48}


%%%%%%%%%%%%%%%%%%%%%%%%%%%%%%%%%%%%%%%%%%%%%%%%%%%%%%%%%%%%%%%%%%%%%%%%%%%%%%%%
%%% Utility functions %%%%%%%%%%%%%%%%%%%%%%%%%%%%%%%%%%%%%%%%%%%%%%%%%%%%%%%%%%

%%% Save space in lists. Use this after the opening of the list %%%%%%%%%%%%%%%%
\newcommand{\compresslist}{
	\setlength{\itemsep}{1pt}
	\setlength{\parskip}{0pt}
	\setlength{\parsep}{0pt}
}

%%%%%%%%%%%%%%%%%%%%%%%%%%%%%%%%%%%%%%%%%%%%%%%%%%%%%%%%%%%%%%%%%%%%%%%%%%%%%%%
%%% Document Start %%%%%%%%%%%%%%%%%%%%%%%%%%%%%%%%%%%%%%%%%%%%%%%%%%%%%%%%%%%%
%%%%%%%%%%%%%%%%%%%%%%%%%%%%%%%%%%%%%%%%%%%%%%%%%%%%%%%%%%%%%%%%%%%%%%%%%%%%%%%

\begin{document}
\typeout{Poster rendering started}

%%% Setting Background Image %%%%%%%%%%%%%%%%%%%%%%%%%%%%%%%%%%%%%%%%%%%%%%%%%%
\background{
%	\begin{tikzpicture}[remember picture,overlay]%
%	\draw (current page.north west)+(-2em,2em) node[anchor=north west]
%	{\includegraphics[height=1.1\textheight]{background}};
%	\end{tikzpicture}
}

%%% General Poster Settings %%%%%%%%%%%%%%%%%%%%%%%%%%%%%%%%%%%%%%%%%%%%%%%%%%%
%%%%%% Eye Catcher, Title, Authors and University Images %%%%%%%%%%%%%%%%%%%%%%
\begin{poster}{
	grid=false,
	% Option is left on true though the eyecatcher is not used. The reason is
	% that we have a bit nicer looking title and author formatting in the headercol
	% this way
	%eyecatcher=false, 
	borderColor=bordercol,
	headerColorOne=headercol2,
	headerColorTwo=headercol2,
	headerFontColor=headerfontcol,
	% Only simple background color used, no shading, so boxColorTwo isn't necessary
	boxColorOne=boxcolor,
	headershape=rounded,
	headerfont=\Large\sf\bf,
	headerFontColor=white,
	textborder=rounded,
	background=user,
	headerborder=open,
  	boxshade=plain,
  	columns=4
}
%%% Eye Cacther %%%%%%%%%%%%%%%%%%%%%%%%%%%%%%%%%%%%%%%%%%%%%%%%%%%%%%%%%%%%%%%
{
	Eye Catcher, empty if option eyecatcher=false - unused
}
%%% Title %%%%%%%%%%%%%%%%%%%%%%%%%%%%%%%%%%%%%%%%%%%%%%%%%%%%%%%%%%%%%%%%%%%%%
{\sf\bf
	Development of a Simple Near-Ground Path Loss Model Verified by Measurements
}
%%% Authors %%%%%%%%%%%%%%%%%%%%%%%%%%%%%%%%%%%%%%%%%%%%%%%%%%%%%%%%%%%%%%%%%%%
{
	\vspace{1em} Kemal Kapetanovic, Mads Gotthardsen, Thomas Jørgensen\\
	{\smaller (kkapet08, mgotth13, tkjj13)@student.aau.dk WCS7 2016}
}
%%% Logo %%%%%%%%%%%%%%%%%%%%%%%%%%%%%%%%%%%%%%%%%%%%%%%%%%%%%%%%%%%%%%%%%%%%%%
{
% The logos are compressed a bit into a simple box to make them smaller on the result
% (Wasn't able to find any bigger of them.)
\setlength\fboxsep{0pt}
\setlength\fboxrule{0.5pt}
	\fbox{
		\begin{minipage}{14em}
%			\includegraphics[width=10em,height=4em]{colbud_logo}
%			\includegraphics[width=4em,height=4em]{elte_logo} \\
%			\includegraphics[width=10em,height=4em]{dynanets_logo}
%			\includegraphics[width=4em,height=4em]{aitia_logo}
		\end{minipage}
	}
}

\headerbox{Problem}{name=Box1,column=0,span=1,row=0}{
\large

In the future there will be used more wireless sensor networks  to different task and many nodes in these networks, can be placed at low heights, where communication between nodes get worse, as the path loss (PL) increases as the multipath waves can no longer be ignored. This will effect the link budget, when designing the antennas.
\begin{center}


\includegraphics[scale=0.8]{pix/poster_cropped.pdf}
\captionof{figure}{The different waves effecting the PL}
\label{fig:name}
\end{center}

}




\headerbox{PL Models}
{name=Box2,column=1,span=2,row=0}{
\large

\begin{minipage}{.45\textwidth}
\textcolor{thomasred}{\textbf{Friss free space PL (FSPL)}:}
\end{minipage}%
\hspace{1cm}
\begin{minipage}{0.45\textwidth}
\textcolor{thomasyellow}{\textbf{Norton surface wave PL (NSPL)}:}
\end{minipage} \\

%\begin{minipage}{.5\textwidth}
%\begin{equation*}
%L_p=\left(\frac{4 \pi d}{\lambda}\right)^2
%\label{simple_friss}
%\end{equation*} 
%\end{minipage}%
%\begin{minipage}{0.5\textwidth}
%\begin{equation*}
%L_p=\left(\frac{d}{\left|\frac{\lambda}{2\pi z}\right|}\right)^4
%\label{surface_wave}
%\end{equation*}
%\end{minipage} \\

\begin{minipage}{0.45\textwidth}
This model only uses the direct wave and do not take any reflections into account. It is not reliable at low heights. 
\end{minipage}%
\hspace{1cm}
\begin{minipage}{0.45\textwidth}
This model only takes into account the surface wave and is not reliable at higher heights, than where the surface wave effects.
\end{minipage} \\

%\vspace{0.3in} 

\begin{minipage}{.45\textwidth}
\textcolor{thomasblue}{\textbf{Approximated two-ray}}\\
\textcolor{thomasblue}{\textbf{ground-reflection PL (ATRPL)}:}
\end{minipage}%
\hspace{1cm}
\begin{minipage}{0.45\textwidth}
\textcolor{thomaspurple}{\textbf{Ground wave PL (GWPL)}:}
\end{minipage} \\



%\begin{minipage}{.5\textwidth}
%\begin{equation*}
%L_{p} = \left(\frac{d^2}{h_t h_r}\right)^2
%\label{two_ray_model}
%\end{equation*}
%\end{minipage}%
%\begin{minipage}{0.5\textwidth}
%\begin{equation*}
%L_p=\left(\frac{4 \pi d}{\lambda}\right)^2 \cdot \Big|1+R\text{e}^{j\Delta}+(1-R)A\text{e}^{j\Delta}\Big|^{-2} 
%\label{ground_wave}
%\end{equation*}
%\end{minipage} \\

\begin{minipage}{.45\textwidth}
This model takes the direct wave and a single reflection into account. This aproximated version of the model works best in the heights between FSPL and NSPL.
\end{minipage}%
\hspace{1cm}
\begin{minipage}{0.45\textwidth}
This model takes into account all the waves, seen on figure 1 and also the reflection and absorbing coefficient of the surface.
\end{minipage} \\

}




\headerbox{Test setup}{name=Box3,column=0,span=1, below=Box1, above=bottom}{
\large
A measurement campaign were designed with given different parameters. By knowing system gains and losses, the PL can be calculated.
For the test setup there where these different parameters;
\begin{itemize}
\item 2 Antenna sets at 858MHz (monopole and rectangular patch)
\item 2 Polarization (horizontal and vertical)
\item 2 Location (parking lot and school gym)
\item 4 different height for the antennas (0.04, 0.14, 0.36 and 2.02 m)
\item 6 distances between antennas (1, 2, 4, 8, 15 and 30 m)
\end{itemize}

In each point, 10 measurements were performed and the mean hereof were found, to lessen the effect of small scale fading.

}



\headerbox{New proposed PL model}{name=Box5,span=2,column=1,row=0, below=Box2, above=bottom}{
\large

\begin{minipage}{.45\textwidth}
*Forklarende tekst*
\begin{equation*}
L_p = \frac{d^4}{h_t^2 h_r^2+h_0^4}
\end{equation*}
\end{minipage}%
\hspace{1cm}
\begin{minipage}{0.45\textwidth}
*Her snakker vi omkring z og fuck up irrenterende den er*
\end{minipage} \\

\begin{minipage}{.45\textwidth}
\begin{tabular}{|l|l|l|}
\hline
\textbf{Models} & \textbf{MSE} & \textbf{Applicability} \\ \hline
FSPL            & 15.95        & 35 \%                  \\ \hline
ATRPL 		    & 141.58       & 65 \%                  \\ \hline %approx.
%TRPL     		& 42.12        & 100 \%                 \\ \hline
GWPL            & 35.49        & 100 \%                 \\ \hline
NSPL            & 230.05       & 30 \%                  \\ \hline
NPPL            & 60.18        & 65 \%                  \\ \hline
\end{tabular}
\end{minipage}%
\hspace{1cm}
\begin{minipage}{0.45\textwidth}
% This file was created by matlab2tikz.
%
%The latest updates can be retrieved from
%  http://www.mathworks.com/matlabcentral/fileexchange/22022-matlab2tikz-matlab2tikz
%where you can also make suggestions and rate matlab2tikz.
%
\definecolor{mycolor1}{rgb}{0.00000,0.44700,0.74100}%
\definecolor{mycolor2}{rgb}{0.85000,0.32500,0.09800}%
\definecolor{mycolor3}{rgb}{0.92900,0.69400,0.12500}%
%
\begin{tikzpicture}

\begin{axis}[%
width=2.51in,
height=1.6in,
at={(2.6in,1in)},
scale only axis,
extra x ticks={2,5,20}, 
extra x tick style={log identify minor tick positions=false},
log ticks with fixed point,
xticklabel style={yshift=-0.5ex},
yticklabel style={xshift=-0.5ex},
xmode=log,
xmin=0.9,
xmax=31,
xlabel=Distance (m),
xminorticks=true,
xmajorgrids,
xminorgrids,
ymin=20,
ymax=100,
ylabel=Path loss (dB),
ymajorgrids,
axis background/.style={fill=white},
legend style={at={(0.03,0.97)},anchor=north west,legend cell align=left,align=left,draw=white!15!black},
legend pos = south east
]
\addplot [color=mycolor1,mark size=2.5pt,only marks,mark=asterisk,mark options={solid}]
  table[row sep=crcr]{%
1	39.176832441447\\
2	50.0236813268286\\
4	61.7627349006033\\
8	69.1568155321866\\
15	80.2784721484843\\
30	81.8463231335614\\
};
\addlegendentry{Measured PL};

\addplot [color=mycolor2,solid]
  table[row sep=crcr]{%
1	45.9420286016976\\
2	57.9832284282568\\
4	70.0244282548161\\
8	82.0656280813753\\
15	92.9856789639248\\
30	105.026878790484\\
};
\addlegendentry{Mean z (0.8122)};

\addplot [color=mycolor3,solid]
  table[row sep=crcr]{%
1	34.2247802230071\\
2	46.2659800495663\\
4	58.3071798761256\\
8	70.3483797026848\\
15	81.2684305852343\\
30	93.3096304117936\\
};
\addlegendentry{Adjusted z (0.4)};

\end{axis}
\end{tikzpicture}%
\end{minipage} \\


}

\headerbox{Comparison}
{name=Box6,span=1,column=3}{
\large
\begin{center}

\input{pix/Models72.tex}

% This file was created by matlab2tikz.
%
%The latest updates can be retrieved from
%  http://www.mathworks.com/matlabcentral/fileexchange/22022-matlab2tikz-matlab2tikz
%where you can also make suggestions and rate matlab2tikz.
%
\definecolor{mycolor1}{rgb}{0.00000,0.44700,0.74100}%
\definecolor{mycolor2}{rgb}{0.85000,0.32500,0.09800}%
\definecolor{mycolor3}{rgb}{0.92900,0.69400,0.12500}%
\definecolor{mycolor4}{rgb}{0.49400,0.18400,0.55600}%
\definecolor{mycolor5}{rgb}{0.46600,0.67400,0.18800}%
%
\begin{tikzpicture}

\begin{axis}[%
width=4.521in,
height=3.566in,
at={(0.758in,0.481in)},
scale only axis,
xmode=log,
xmin=0,
xmax=31,
xminorticks=true,
xmajorgrids,
xminorgrids,
ymin=-100,
ymax=-20,
ymajorgrids,
axis background/.style={fill=white},
title style={font=\bfseries},
title={Trace 1}
]
\addplot [color=mycolor1,mark size=2.5pt,only marks,mark=asterisk,mark options={solid},forget plot]
  table[row sep=crcr]{%
1	-39.176832441447\\
2	-50.0236813268286\\
4	-61.7627349006033\\
8	-69.1568155321866\\
15	-80.2784721484843\\
30	-81.8463231335614\\
};
\addplot [color=mycolor2,solid,forget plot]
  table[row sep=crcr]{%
1	-31.1115179430228\\
1.29292929292929	-33.3430134440292\\
1.58585858585859	-35.1168070992565\\
1.87878787878788	-36.5890729354301\\
2.17171717171717	-37.8475832493839\\
2.46464646464646	-38.9466105778464\\
2.75757575757576	-39.9220669918869\\
3.05050505050505	-40.7989529102148\\
3.34343434343434	-41.5953739265862\\
3.63636363636364	-42.3248640664175\\
3.92929292929293	-42.9978060775859\\
4.22222222222222	-43.6223396865725\\
4.51515151515152	-44.2049645137105\\
4.80808080808081	-44.7509531054816\\
5.1010101010101	-45.264641613445\\
5.39393939393939	-45.7496391916429\\
5.68686868686869	-46.2089819480987\\
5.97979797979798	-46.6452481855302\\
6.27272727272727	-47.0606460546034\\
6.56565656565657	-47.4570811839289\\
6.85858585858586	-47.8362095366818\\
7.15151515151515	-48.1994792048672\\
7.44444444444444	-48.5481638082528\\
7.73737373737374	-48.8833894437239\\
8.03030303030303	-49.2061566242012\\
8.32323232323232	-49.5173582850141\\
8.61616161616162	-49.8177946744222\\
8.90909090909091	-50.1081857537082\\
9.2020202020202	-50.3891815905317\\
9.49494949494949	-50.6613711230658\\
9.78787878787879	-50.9252895920871\\
10.0808080808081	-51.1814248768192\\
10.3737373737374	-51.4302229230174\\
10.6666666666667	-51.6720924150277\\
10.959595959596	-51.9074088147627\\
11.2525252525253	-52.136517867826\\
11.5454545454545	-52.3597386589774\\
11.8383838383838	-52.5773662847132\\
12.1313131313131	-52.7896741991299\\
12.4242424242424	-52.9969162798597\\
12.7171717171717	-53.199328653229\\
13.010101010101	-53.3971313115477\\
13.3030303030303	-53.5905295503075\\
13.5959595959596	-53.7797152488309\\
13.8888888888889	-53.9648680143974\\
14.1818181818182	-54.1461562069475\\
14.4747474747475	-54.3237378590187\\
14.7676767676768	-54.4977615035086\\
15.0606060606061	-54.6683669201317\\
15.3535353535354	-54.8356858099672\\
15.6464646464646	-54.9998424062559\\
15.9393939393939	-55.1609540285398\\
16.2323232323232	-55.3191315863387\\
16.5252525252525	-55.4744800377779\\
16.8181818181818	-55.6270988079186\\
17.1111111111111	-55.7770821709656\\
17.4040404040404	-55.9245196000324\\
17.6969696969697	-56.069496087713\\
17.989898989899	-56.2120924403367\\
18.2828282828283	-56.3523855484555\\
18.5757575757576	-56.4904486358333\\
18.8686868686869	-56.6263514889533\\
19.1616161616162	-56.760160668845\\
19.4545454545455	-56.8919397068421\\
19.7474747474747	-57.0217492857095\\
20.040404040404	-57.149647407435\\
20.3333333333333	-57.2756895488449\\
20.6262626262626	-57.3999288060896\\
20.9191919191919	-57.5224160289408\\
21.2121212121212	-57.6431999457502\\
21.5050505050505	-57.7623272798382\\
21.7979797979798	-57.8798428580096\\
22.0909090909091	-57.9957897118245\\
22.3838383838384	-58.1102091721996\\
22.6767676767677	-58.2231409578586\\
22.969696969697	-58.3346232581061\\
23.2626262626263	-58.4446928103564\\
23.5555555555556	-58.5533849728113\\
23.8484848484848	-58.6607337926463\\
24.1414141414141	-58.7667720700345\\
24.4343434343434	-58.8715314183094\\
24.7272727272727	-58.9750423205423\\
25.020202020202	-59.0773341827885\\
25.3131313131313	-59.1784353842344\\
25.6060606060606	-59.2783733244589\\
25.8989898989899	-59.3771744680074\\
26.1919191919192	-59.4748643864588\\
26.4848484848485	-59.5714677981531\\
26.7777777777778	-59.6670086057337\\
27.0707070707071	-59.7615099316476\\
27.3636363636364	-59.8549941517351\\
27.6565656565657	-59.9474829270312\\
27.9494949494949	-60.0389972338908\\
28.2424242424242	-60.1295573925447\\
28.5353535353535	-60.2191830941809\\
28.8282828282828	-60.3078934266443\\
29.1212121212121	-60.3957068988359\\
29.4141414141414	-60.4826414638918\\
29.7070707070707	-60.5687145412132\\
30	-60.653943037416\\
};
\addplot [color=mycolor3,solid,forget plot]
  table[row sep=crcr]{%
1	-54.6612617768166\\
1.29292929292929	-59.1242527788293\\
1.58585858585859	-62.6718400892839\\
1.87878787878788	-65.6163717616312\\
2.17171717171717	-68.1333923895388\\
2.46464646464646	-70.3314470464637\\
2.75757575757576	-72.2823598745448\\
3.05050505050505	-74.0361317112006\\
3.34343434343434	-75.6289737439433\\
3.63636363636364	-77.087954023606\\
3.92929292929293	-78.4338380459428\\
4.22222222222222	-79.6829052639159\\
4.51515151515152	-80.848154918192\\
4.80808080808081	-81.9401321017343\\
5.1010101010101	-82.967509117661\\
5.39393939393939	-83.9375042740568\\
5.68686868686869	-84.8561897869684\\
5.97979797979798	-85.7287222618313\\
6.27272727272727	-86.5595179999777\\
6.56565656565657	-87.3523882586288\\
6.85858585858586	-88.1106449641346\\
7.15151515151515	-88.8371843005053\\
7.44444444444444	-89.5345535072766\\
7.73737373737374	-90.2050047782187\\
8.03030303030303	-90.8505391391734\\
8.32323232323232	-91.4729424607992\\
8.61616161616162	-92.0738152396155\\
8.90909090909091	-92.6545973981873\\
9.2020202020202	-93.2165890718345\\
9.49494949494949	-93.7609681369025\\
9.78787878787879	-94.2888050749451\\
10.0808080808081	-94.8010756444094\\
10.3737373737374	-95.2986717368057\\
10.6666666666667	-95.7824107208263\\
10.959595959596	-96.2530435202965\\
11.2525252525253	-96.7112616264229\\
11.5454545454545	-97.1577032087258\\
11.8383838383838	-97.5929584601974\\
12.1313131313131	-98.0175742890308\\
12.4242424242424	-98.4320584504904\\
12.7171717171717	-98.8368831972291\\
13.010101010101	-99.2324885138663\\
13.3030303030303	-99.6192849913859\\
13.5959595959596	-99.9976563884329\\
13.8888888888889	-100.367961919566\\
14.1818181818182	-100.730538304666\\
14.4747474747475	-101.085701608808\\
14.7676767676768	-101.433748897788\\
15.0606060606061	-101.774959731034\\
15.3535353535354	-102.109597510705\\
15.6464646464646	-102.437910703283\\
15.9393939393939	-102.760133947851\\
16.2323232323232	-103.076489063448\\
16.5252525252525	-103.387185966327\\
16.8181818181818	-103.692423506608\\
17.1111111111111	-103.992390232702\\
17.4040404040404	-104.287265090836\\
17.6969696969697	-104.577218066197\\
17.989898989899	-104.862410771444\\
18.2828282828283	-105.142996987682\\
18.5757575757576	-105.419123162438\\
18.8686868686869	-105.690928868678\\
19.1616161616162	-105.958547228461\\
19.4545454545455	-106.222105304455\\
19.7474747474747	-106.48172446219\\
20.040404040404	-106.737520705641\\
20.3333333333333	-106.989604988461\\
20.6262626262626	-107.23808350295\\
20.9191919191919	-107.483057948653\\
21.2121212121212	-107.724625782271\\
21.5050505050505	-107.962880450447\\
21.7979797979798	-108.19791160679\\
22.0909090909091	-108.42980531442\\
22.3838383838384	-108.65864423517\\
22.6767676767677	-108.884507806488\\
22.969696969697	-109.107472406983\\
23.2626262626263	-109.327611511484\\
23.5555555555556	-109.544995836394\\
23.8484848484848	-109.759693476064\\
24.1414141414141	-109.97177003084\\
24.4343434343434	-110.18128872739\\
24.7272727272727	-110.388310531855\\
25.020202020202	-110.592894256348\\
25.3131313131313	-110.79509665924\\
25.6060606060606	-110.994972539689\\
25.8989898989899	-111.192574826786\\
26.1919191919192	-111.387954663689\\
26.4848484848485	-111.581161487077\\
26.7777777777778	-111.772243102238\\
27.0707070707071	-111.961245754066\\
27.3636363636364	-112.148214194241\\
27.6565656565657	-112.333191744833\\
27.9494949494949	-112.516220358553\\
28.2424242424242	-112.69734067586\\
28.5353535353535	-112.876592079133\\
28.8282828282828	-113.05401274406\\
29.1212121212121	-113.229639688443\\
29.4141414141414	-113.403508818555\\
29.7070707070707	-113.575654973197\\
30	-113.746111965603\\
};
\addplot [color=mycolor4,solid,forget plot]
  table[row sep=crcr]{%
1	-44.9214537765832\\
1.29292929292929	-49.050887223414\\
1.58585858585859	-52.3850957227753\\
1.87878787878788	-55.1813933158306\\
2.17171717171717	-57.5894450858098\\
2.46464646464646	-59.7040212694623\\
2.75757575757576	-61.5889394607992\\
3.05050505050505	-63.2892290586399\\
3.34343434343434	-64.8378512938563\\
3.63636363636364	-66.2596600128227\\
3.92929292929293	-67.5738601151063\\
4.22222222222222	-68.7955994809671\\
4.51515151515152	-69.9370368618302\\
4.80808080808081	-71.0080800079665\\
5.1010101010101	-72.0169091509038\\
5.39393939393939	-72.970356641531\\
5.68686868686869	-73.8741877133422\\
5.97979797979798	-74.7333117479744\\
6.27272727272727	-75.5519437159365\\
6.56565656565657	-76.333729260202\\
6.85858585858586	-77.0818428258175\\
7.15151515151515	-77.7990655182941\\
7.44444444444444	-78.4878475170174\\
7.73737373737374	-79.1503585803853\\
8.03030303030303	-79.7885292691858\\
8.32323232323232	-80.4040848627714\\
8.61616161616162	-80.9985734692596\\
8.90909090909091	-81.5733894830492\\
9.2020202020202	-82.1297932842206\\
9.49494949494949	-82.668927879939\\
9.78787878787879	-83.1918330403857\\
10.0808080808081	-83.699457368673\\
10.3737373737374	-84.1926686568157\\
10.6666666666667	-84.6722628117708\\
10.959595959596	-85.1389715821175\\
11.2525252525253	-85.5934692737141\\
11.5454545454545	-86.0363786090357\\
11.8383838383838	-86.4682758579694\\
12.1313131313131	-86.8896953461372\\
12.4242424242424	-87.3011334292265\\
12.7171717171717	-87.7030520074711\\
13.010101010101	-88.0958816426928\\
13.3030303030303	-88.4800243306433\\
13.5959595959596	-88.8558559734019\\
13.8888888888889	-89.223728589948\\
14.1818181818182	-89.5839722974902\\
14.4747474747475	-89.9368970915016\\
14.7676767676768	-90.2827944485156\\
15.0606060606061	-90.6219387724419\\
15.3535353535354	-90.9545887023927\\
15.6464646464646	-91.2809882976234\\
15.9393939393939	-91.6013681131972\\
16.2323232323232	-91.9159461782433\\
16.5252525252525	-92.2249288872049\\
16.8181818181818	-92.5285118132048\\
17.1111111111111	-92.8268804515537\\
17.4040404040404	-93.1202109004784\\
17.6969696969697	-93.4086704853281\\
17.989898989899	-93.6924183317962\\
18.2828282828283	-93.9716058930773\\
18.5757575757576	-94.2463774353278\\
18.8686868686869	-94.5168704853298\\
19.1616161616162	-94.7832162438333\\
19.4545454545455	-95.0455399676875\\
19.7474747474747	-95.3039613235475\\
20.040404040404	-95.5585947156543\\
20.3333333333333	-95.8095495899402\\
20.6262626262626	-96.0569307164769\\
20.9191919191919	-96.3008384520958\\
21.2121212121212	-96.5413689848235\\
21.5050505050505	-96.7786145616295\\
21.7979797979798	-97.0126637008234\\
22.0909090909091	-97.2436013903413\\
22.3838383838384	-97.4715092730185\\
22.6767676767677	-97.6964658198685\\
22.969696969697	-97.9185464922848\\
23.2626262626263	-98.1378238940042\\
23.5555555555556	-98.3543679135994\\
23.8484848484848	-98.5682458581985\\
24.1414141414141	-98.7795225790731\\
24.4343434343434	-98.9882605896812\\
24.7272727272727	-99.1945201767008\\
25.020202020202	-99.3983595045488\\
25.3131313131313	-99.5998347138414\\
25.6060606060606	-99.7990000142057\\
25.8989898989899	-99.9959077718343\\
26.1919191919192	-100.190608592132\\
26.4848484848485	-100.383151397784\\
26.7777777777778	-100.573583502539\\
27.0707070707071	-100.761950681\\
27.3636363636364	-100.948297234668\\
27.6565656565657	-101.132666054478\\
27.9494949494949	-101.315098680055\\
28.2424242424242	-101.495635355893\\
28.5353535353535	-101.674315084635\\
28.8282828282828	-101.851175677652\\
29.1212121212121	-102.02625380307\\
29.4141414141414	-102.199585031394\\
29.7070707070707	-102.371203878887\\
30	-102.541143848821\\
};
\addplot [color=mycolor5,solid,forget plot]
  table[row sep=crcr]{%
1	-45.2757493241887\\
1.29292929292929	-49.7228898745648\\
1.58585858585859	-53.262521493136\\
1.87878787878788	-56.2025058764576\\
2.17171717171717	-58.7166869357065\\
2.46464646464646	-60.912850849718\\
2.75757575757576	-62.8624418401923\\
3.05050505050505	-64.6152535121936\\
3.34343434343434	-66.2073762348812\\
3.63636363636364	-67.665803769948\\
3.92929292929293	-69.0112538998\\
4.22222222222222	-70.2599742986074\\
4.51515151515152	-71.4249423794752\\
4.80808080808081	-72.5166878414888\\
5.1010101010101	-73.5438718794334\\
5.39393939393939	-74.5137046228799\\
5.68686868686869	-75.4322521586162\\
5.97979797979798	-76.3046664254687\\
6.27272727272727	-77.135360121141\\
6.56565656565657	-77.9281416843724\\
6.85858585858586	-78.6863208115117\\
7.15151515151515	-79.41279190354\\
7.44444444444444	-80.110100760591\\
7.73737373737374	-80.7804984041527\\
8.03030303030303	-81.4259848975564\\
8.32323232323232	-82.0483453152401\\
8.61616161616162	-82.6491794904726\\
8.90909090909091	-83.2299267897558\\
9.2020202020202	-83.7918868793997\\
9.49494949494949	-84.3362372379378\\
9.78787878787879	-84.864048007723\\
10.0808080808081	-85.3762946565475\\
10.3737373737374	-85.8738688257092\\
10.6666666666667	-86.3575876675597\\
10.959595959596	-86.8282019180994\\
11.2525252525253	-87.2864029048315\\
11.5454545454545	-87.7328286540697\\
11.8383838383838	-88.1680692330933\\
12.1313131313131	-88.5926714393722\\
12.4242424242424	-89.0071429303434\\
12.7171717171717	-89.4119558719659\\
13.010101010101	-89.8075501718137\\
13.3030303030303	-90.1943363522118\\
13.5959595959596	-90.5726981104615\\
13.8888888888889	-90.9429946061772\\
14.1818181818182	-91.3055625099107\\
14.4747474747475	-91.6607178423471\\
14.7676767676768	-92.0087576292505\\
15.0606060606061	-92.3499613938723\\
15.3535353535354	-92.684592505612\\
15.6464646464646	-93.0128994012286\\
15.9393939393939	-93.3351166927899\\
16.2323232323232	-93.651466174734\\
16.5252525252525	-93.9621577408714\\
16.8181818181818	-94.2673902208208\\
17.1111111111111	-94.5673521442283\\
17.4040404040404	-94.8622224401226\\
17.6969696969697	-95.1521710779052\\
17.989898989899	-95.4373596557222\\
18.2828282828283	-95.717941941319\\
18.5757575757576	-95.9940643699092\\
18.8686868686869	-96.2658665030932\\
19.1616161616162	-96.5334814524268\\
19.4545454545455	-96.7970362708578\\
19.7474747474747	-97.0566523149128\\
20.040404040404	-97.3124455802169\\
20.3333333333333	-97.5645270126694\\
20.6262626262626	-97.8130027973644\\
20.9191919191919	-98.0579746271396\\
21.2121212121212	-98.2995399524521\\
21.5050505050505	-98.5377922141197\\
21.7979797979798	-98.7728210603168\\
22.0909090909091	-99.0047125490867\\
22.3838383838384	-99.233549337516\\
22.6767676767677	-99.4594108586096\\
22.969696969697	-99.6823734868158\\
23.2626262626263	-99.9025106930627\\
23.5555555555556	-100.119893190094\\
23.8484848484848	-100.334589068825\\
24.1414141414141	-100.546663926372\\
24.4343434343434	-100.756180986366\\
24.7272727272727	-100.963201212092\\
25.020202020202	-101.167783412968\\
25.3131313131313	-101.369984344831\\
25.6060606060606	-101.569858804446\\
25.8989898989899	-101.767459718646\\
26.1919191919192	-101.962838228456\\
26.4848484848485	-102.156043768541\\
26.7777777777778	-102.347124142282\\
27.0707070707071	-102.53612559277\\
27.3636363636364	-102.72309286998\\
27.6565656565657	-102.908069294363\\
27.9494949494949	-103.091096817097\\
28.2424242424242	-103.27221607719\\
28.5353535353535	-103.451466455638\\
28.8282828282828	-103.628886126821\\
29.1212121212121	-103.804512107298\\
29.4141414141414	-103.978380302156\\
29.7070707070707	-104.150525549075\\
30	-104.320981660217\\
};
\end{axis}
\end{tikzpicture}%

\end{center}
\captionof{figure}{Comparison between the different PL model. At the top, the Tx and Rx height are 0.14 and 2.02 m and at the bottom, both are 0.04 m \\
The points are the measured PL, The red is FSPL, the blue is ATRPL, the yellow is NSPL, the purple is GWPL and the black is the NPPL.}


}




\headerbox{Acknowledgements}{name=Box4,column=3,below=Box6,span=1}{
\large
The authors would like to thank Vendelbo hallen for providing access to their facilities during the measurement campaign.
}



\headerbox{Reference}
{name=Box7,span=1,column=3,below=Box4,above=bottom}{
\large
[1] K. Bullington, “Radio Propagation at Frequencies Above 30 Megacycles,” \newline
[2] P. K. Chong and D. Kim, “Surface-Level Path Loss Modelling for Sensor Networks in Flat and Irregular Terrain,” \newline
}



\end{poster}
\end{document}
